\documentclass[12pt]{article}

\usepackage{a4}  
\usepackage{color}
\usepackage{amssymb}
\usepackage{amsmath}
\usepackage[utf8]{inputenc}
\usepackage{faktor}
 
\newcommand{\corr}[1]{\textcolor{red}{#1}}
\newcommand{\QED}{\begin{flushright} $\square$ \end{flushright}}
\newcommand{\df}{\enspace\Longrightarrow\enspace}
\newcommand{\koeff}[2]{\begin{pmatrix}#1 \\ #2\end{pmatrix}}
\newcommand{\vek}[2]{\begin{pmatrix}#1 \\ $\vdots$ \\ #2\end{pmatrix}}
\newcommand{\Char}{\operatorname{char}}
\newcommand{\isIdeal}{\trianglelefteq}
\newcommand{\ann}{\operatorname{ann}}
\newcommand{\ideal}[1]{\langle#1\rangle}
\newcommand{\N}{\operatorname{N}}
\newcommand{\enorm}{\operatorname{d}}
\newcommand{\gdw}{\;\Longleftrightarrow\;}
\newcommand{\abs}[1]{\vert #1\vert}
\newcommand{\ggT}{\operatorname{ggT}}
\newcommand{\kgV}{\operatorname{kgV}}
\newcommand{\f}{\operatorname{f}}
\renewcommand{\d}{\operatorname{d}}

\newcommand{\aal}{a_{\alpha}}
\newcommand{\ab}{a_{\beta}}
\newcommand{\ba}{b_{\alpha}}
\newcommand{\bb}{b_{\beta}}



\begin{document}
\section*{Abgabe Analysis IIb, Blatt 02}

Studierende(r): Weerts, Steffen, steffen.weerts@uni-oldenburg.de

\subsection*{Aufgabe 1}
\begin{enumerate}
	\item[(a)] Sei $f:\mathbb{R}^2\rightarrow\mathbb{R},\koeff{x}{y}\mapsto x$. \\
	Zu zeigen: $M\subset\mathbb{R}^2$ offen $\df\f(M)\subset\mathbb{R}$ offen. \\
	Es gilt:
	\begin{align*}
		&M\subset\mathbb{R}^2\text{ offen} \\
		\df &\forall x\in M\exists\varepsilon>0:B_{\epsilon}(x)\subset M \\
		\df &\forall x\in M\exists\varepsilon>0:\left\{ x\in\mathbb{R}^2:\d(x,a)<\varepsilon\right\} \\	
		\df &\forall x\in M\exists\varepsilon>0:\left\{ x\in\mathbb{R}^2:\sqrt{(x_1-a_1)^2+(x_2-a_2)^2}<\varepsilon\right\}
	\end{align*}
	Außerdem gilt: $$\varepsilon>\sqrt{(x_1-a_1)^2+(x_2-a_2)^2}>\sqrt{(x_1-a_1)^2}=\vert x_1-a_1\vert$$
	Also folgt:
	\begin{align*}
		&\forall x\in M\exists\varepsilon>0:\left\{a\in\mathbb{R}^2:\vert x_1-a_1\vert<\varepsilon\right\} \\
		\df &\forall \koeff{x_1}{x_2}\in M\exists\varepsilon>0:\left\{\koeff{a_1}{a_2}\in\mathbb{R}^2:\vert x_1-a_1\vert<\varepsilon\right\} \\
		\df &\forall\f(\koeff{x_1}{x_2})\in\f(M)\exists\varepsilon>0:\left\{\f(\koeff{a_1}{a_2})\in\mathbb{R}:\vert x_1-a_1\vert<\varepsilon\right\} \\
		\df &\forall\f(\koeff{x_1}{x_2})\in\f(M)\exists\varepsilon>0:B_{\varepsilon}\left(\f\left(\koeff{a_1}{a_2}\right)\right)<\varepsilon \\
		\df &\f(M)\text{ ist offen in }\mathbb{R}.
	\end{align*}
	\QED
	
	\item[(b)] Sei $M\subset\mathbb{R}^2$ abgeschlossen. \\
	Zu zeigen: $\f(M)\subset\mathbb{R}$ abgeschlossen. \\
	Es gilt:
	\begin{align*}
		&M\subset\mathbb{R}^2\text{ abgeschlossen} \\
		\df &\mathbb{R}^2\setminus M\text{ offen} \\
		\overset{(a)}{\df} &\mathbb{R}\setminus\f(M)\text{ offen in }\mathbb{R} \\
		\df &\f(M)\text{ abgeschlossen in }\mathbb{R}.
	\end{align*}
	\QED
\end{enumerate}

\subsection*{Aufgabe 2}
\begin{enumerate}
	\item[(a)] Sei $f_i:\mathbb{R}^n\rightarrow\mathbb{R}, f_i(x_1,\cdots,x_n)=x_i$ für $i=1,\cdots,n$. \\
	Zu zeigen: $f$ stetig. \\
	Es gilt:
	\begin{align*}
		&\vek{x_{n_1}}{x_{n_n}}\rightarrow\vek{a_1}{a_n} \\
		\df &x_{n_1}\rightarrow a_1,\cdots,x_{n_n}\rightarrow a_n \\
		\df &f_i(x_n)\rightarrow f_i(a).
	\end{align*}
	\QED
	
	\item[(b)] Fehlt.
\end{enumerate}

\subsection*{Aufgabe 3}
\begin{enumerate}
	\item[(a)] Fehlt.
	
	\item[(b)] Fehlt.
	
	\item[(c)] Fehlt.
\end{enumerate}

\subsection*{Aufgabe 4}
\begin{enumerate}
	\item[(a)] Fehlt.
	
	\item[(b)] Fehlt.
\end{enumerate}

\end{document}