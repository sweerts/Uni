\documentclass[12pt]{article}

\usepackage{a4}  
\usepackage{color}
\usepackage{amssymb}
\usepackage{amsmath}
\usepackage[utf8]{inputenc}
\usepackage{faktor}
 
\newcommand{\corr}[1]{\textcolor{red}{#1}}
\newcommand{\QED}{\begin{flushright} $\square$ \end{flushright}}
\newcommand{\df}{\enspace\Longrightarrow\enspace}
\newcommand{\koeff}[2]{\begin{pmatrix}#1 \\ #2\end{pmatrix}}
\newcommand{\koefff}[3]{\begin{pmatrix}#1 \\ #2 \\ #3\end{pmatrix}}
\newcommand{\Char}{\operatorname{char}}
\newcommand{\isIdeal}{\trianglelefteq}
\newcommand{\ann}{\operatorname{ann}}
\newcommand{\ideal}[1]{\langle#1\rangle}
\newcommand{\N}{\operatorname{N}}
\newcommand{\enorm}{\operatorname{d}}
\newcommand{\gdw}{\;\Longleftrightarrow\;}
\newcommand{\abs}[1]{\vert #1\vert}
\newcommand{\ggT}{\operatorname{ggT}}
\newcommand{\kgV}{\operatorname{kgV}}
\newcommand{\grad}{\operatorname{deg}}
\newcommand{\LC}{\operatorname{LC}}
\newcommand{\determinante}{\operatorname{det}}

\newcommand{\aal}{a_{\alpha}}
\newcommand{\ab}{a_{\beta}}
\newcommand{\ba}{b_{\alpha}}
\newcommand{\bb}{b_{\beta}}



\begin{document}
\section*{Abgabe Analysis IIb, Blatt 5}

Studierende(r): Weerts, Steffen, steffen.weerts@uni-oldenburg.de

\subsection*{Aufgabe 1}
Fehlt.

\subsection*{Aufgabe 2}
\begin{enumerate}
	\item[(a)] Fehlt.
	
	\item[(b)] Sei $f:\mathbb{R}^2\rightarrow\mathbb{R},f(x,y)=(y^2-x)\cdot(y^2-2x)$. \\
	Zu zeigen: $f$ hat kein lokales Minimum in $\koeff{0}{0}$. \\
	Es gilt: $$f(x,y)=(y^2-x)\cdot(y^2-2x)=y^4-2xy-xy^2+2x^2.$$
	Außerdem gilt:
	\begin{align*}
		&\frac{\partial f}{\partial x}(a)=-2y-y^2+4x, \\
		&\frac{\partial f}{\partial y}(a)=4y^3-2x-2xy, \\
		\df &\nabla f(a)=\koeff{\frac{\partial f}{\partial x}}{\frac{\partial f}{\partial y}}=\koeff{4x-y^2-2y}{4y^3-2xy-2x}.
	\end{align*}
	Für $a=\koeff{0}{0}$ ergibt sich: $$\nabla f(a)=\koeff{\frac{\partial f}{\partial x}}{\frac{\partial f}{\partial y}}=\koeff{4\cdot0-0^2-2\cdot0}{4\cdot0^3-2\cdot0\cdot0-2\cdot0}=\koeff{0}{0}.$$
	Zu zeigen: $H_f(a)$ nicht positiv definit. Es gilt:
	\begin{align*}
		\frac{\partial^2 f}{\partial x^2}(a)&=4, \\
		\frac{\partial^2 f}{\partial y^2}(a)&=12y^2-2x, \\
		\frac{\partial^2 f}{\partial x\partial y}(a)&=-2y-2=\frac{\partial^2 f}{\partial y\partial x}(a).
		\df H_f(a)=\begin{pmatrix}4 & -2y-2 \\ -2y-2 & 12y^2-2x\end{pmatrix}.
	\end{align*}
	Für $a=\koeff{0}{0}$ ergibt sich: $$H_f(a)=\begin{pmatrix}4 & -2\cdot0-2 \\ -2\cdot0-2 & 12\cdot0^2-2\cdot0\end{pmatrix}=\begin{pmatrix}4 & -2 \\ -2 & 0\end{pmatrix}.$$
	Dabei gilt:
	\begin{align*}
		&\determinante(4)=4, \\
		&\determinante(\begin{pmatrix}4 & -2 \\ -2 & 0\end{pmatrix})=4\cdot0-(-2)\cdot(-2)=-4.
		\df &H_f(\koeff{0}{0})\text{ nicht positiv definit}.
	\end{align*}
	Da $H_f(\koeff{0}{0})$ nicht positiv definit ist, besitzt $f$ in $\koeff{0}{0}$ kein lokales Minimum.
	\QED
\end{enumerate}

\subsection*{Aufgabe 3}
Fehlt.

\subsection*{Aufgabe 4}
\begin{enumerate}
	\item[(a)] Fehlt.
	
	\item[(b)] Fehlt.
\end{enumerate}

\subsection*{Aufgabe 5}
\begin{enumerate}
	\item[(a)] Sei $f:\mathbb{R}^2\rightarrow\mathbb{R},f(x,y)=2x^3+xy^2+5x^2+y^2$. \\
	Zu zeigen: $f$ hat ein lokales Maximum bei $\koeff{-\frac{5}{3}}{0}$ und ein lokales Minimum bei $\koeff{0}{0}$. \\
	Es gilt:
	\begin{align*}
		\frac{\partial f}{\partial x}(a)&=6x^2+y^2+10x. \\
		\frac{\partial f}{\partial y}(a)&=2xy+2y. \\
		\df \nabla f(a)&=\koeff{6x^2+y^2+10x}{2xy+2y}.
	\end{align*}
	Außerdem gilt:
	$$\nabla f(a)=0\gdw\text{(I)}\,6x^2+y^2+10x=0\text{ und (II)}\,2xy+2y=0.$$
	\begin{enumerate}
		\item[(II):] $$2xy+2y=0\gdw x=-1\text{ oder }y=0.$$
		
		\item[(I):] \underline{$x=-1$:} \\
		Es gilt:
		\begin{align*}
			&\,6\cdot(-1)^2+y^2+10\cdot(-1)=0 \\
			\gdw&\, y^2-4=0 \\
			\gdw&\, y^2=4 \\
			\gdw&\, y=\pm2.
		\end{align*}
		
		\item[(I):] \underline{$y=0$:} \\
		Es gilt:
		\begin{align*}
			&6x^2+0^2+10x=0 \\
			\gdw&\,x^2+\frac{5}{3}x=0 \\
			\gdw&\,x^2+2\cdot\frac{5}{6}x+\left(\frac{5}{6}\right)^2-\left(\frac{5}{6}\right)^2=0 \\
			\gdw&\,\left(x+\frac{5}{6}\right)^2-\left(\frac{5}{6}\right)^2=0 \\
			\gdw&\,\left(x+\frac{5}{6}\right)^2=\left(\frac{5}{6}\right)^2 \\
			\gdw&\,x+\frac{5}{6}=\pm\frac{5}{6} \\
			\gdw&\,x=0\text{ oder }x=-\frac{5}{3}.
		\end{align*}
	\end{enumerate}
	$\df\nabla f(a)=0\gdw a=\koeff{-1}{\pm2}\text{ oder }a=\koeff{0}{0}\text{ oder }a=\koeff{-\frac{5}{3}}{0}.$
	
	Außerdem gilt:
	\begin{align*}
		\frac{\partial^2f}{\partial x^2}(a)&=12x+10. \\
		\frac{\partial^2f}{\partial y^2}(a)&=2x+2. \\
		\frac{\partial^2f}{\partial x\partial y}(a)&=2y=\frac{\partial^2f}{\partial y\partial x}(a).
	\end{align*}
	$\df H_f(a)=\begin{pmatrix}\frac{\partial^2f}{\partial x^2}(a) & \frac{\partial^2f}{\partial x\partial y}(a) \\ \frac{\partial^2f}{\partial y\partial x}(a) & \frac{\partial^2f}{\partial y^2}(a)\end{pmatrix}=\begin{pmatrix}12x+12 & 2y \\ 2y & 2x+2\end{pmatrix}.$
	
	Teste Kandidaten für Extrema:
	\begin{enumerate}
		\item[\underline{$a=\koeff{-1}{2}$:}] Es gilt:
		\begin{align*}
			&\determinante(12\cdot(-1)+10)=-2, \\
			&\determinante(\begin{pmatrix}12\cdot(-1)+10 & 2\cdot2 \\ 2\cdot2 & 2\cdot(-1)+2\end{pmatrix})=(-2)\cdot0-4\cdot4=-16. \\
			\overset{\text{Sylvester-Krit.}}{\df} &\;H_f(\koeff{-1}{2})\text{ weder positiv noch negativ definit}.
		\end{align*}
		
		\item[\underline{$a=\koeff{-1}{-2}$:}] Es gilt:
		\begin{align*}
			&\determinante(12\cdot(-1)+10)=-2, \\
			&\determinante(\begin{pmatrix}12\cdot(-1)+10 & 2\cdot(-2) \\ 2\cdot(-2) & 2\cdot(-1)+2\end{pmatrix})=(-2)\cdot0-(-4)\cdot(-4)=-16. \\
			\overset{\text{Sylvester-Krit.}}{\df} &\;H_f(\koeff{-1}{-2})\text{ weder positiv noch negativ definit}.
		\end{align*}
		
		\item[\underline{$a=\koeff{0}{0}$:}] Es gilt:
		\begin{align*}
			&\determinante(12\cdot(0)+10)=10, \\
			&\determinante(\begin{pmatrix}12\cdot0+10 & 2\cdot0 \\ 2\cdot0 & 2\cdot0+2\end{pmatrix})=(10)\cdot2-0\cdot0=20. \\
			\overset{\text{Sylvester-Krit.}}{\df} &\;H_f(\koeff{0}{0})\text{ positiv definit}.
		\end{align*}
		
		\item[\underline{$a=\koeff{-\frac{5}{3}}{0}$:}] Es gilt:
		\begin{align*}
			&\determinante(12\cdot(-\frac{5}{3})+10)=-12, \\
			&\determinante(\begin{pmatrix}12\cdot(-\frac{5}{3})+10 & 2\cdot0 \\ 2\cdot0 & 2\cdot(-\frac{5}{3})+2\end{pmatrix})=(-10)\cdot\left(-\frac{4}{3}\right)-0=\frac{40}{3}. \\
			\overset{\text{Sylvester-Krit.}}{\df} &\;H_f(\koeff{-\frac{5}{3}}{0})\text{ negativ definit}.
		\end{align*}
	\end{enumerate}
	Insgesamt ergibt sich nach Satz $2.10$, dass $f$ an der Stelle $\koeff{0}{0}$ ein lokales Minimum besitzt und an der Stelle $\koeff{-\frac{5}{3}}{0}$ ein lokales Maximum besitzt. An den Stellen $\koeff{-1}{2}$ und $\koeff{-1}{-2}$ besitzt $f$ keine Extrema.
	\QED
	
	\item[(b)] Sei $f:\mathbb{R}^3\rightarrow\mathbb{R},f(x,y,z)=3x^3+y^2+z^2+6xy-2z+1$. \\
	Zu zeigen: Behauptung. \\
	Es gilt:
	\begin{align*}
		&\frac{\partial f}{\partial x}(a)=9x^2+6y, \\
		&\frac{\partial f}{\partial y}(a)=2y+6x, \\
		&\frac{\partial f}{\partial z}(a)=2z-2. \\
		\df &\nabla f(a)=\koefff{9x^2+6y}{2y+6x}{2z-2}.
	\end{align*}
	Außerdem gilt: $$\nabla f(a)=0\gdw\begin{matrix}\;\text{(I)}\quad9x^2+6y \\ \text{(II)}\quad2y+6x \\ \text{(III)}\quad2z-2y\end{matrix}$$
	\begin{enumerate}
		\item[(III):] $$2z-2=0\gdw z=1.$$
		
		\item[(II):] $$2y+6x=0\gdw 3x+y=0\gdw y=-3x.$$
		
		\item[(I):]
		\begin{align*}
			&9x^2+6y=0 \\
			\overset{\text{(II)}}{\gdw}&9x^2+6\cdot(-3x)=0 \\
			\gdw&9x^2-18x=0 \\
			\gdw&x^2-2x=0 \\
			\gdw&x^2-2x+1^2-1^2=0 \\
			\gdw&(x-1)^2=1^2 \\
			\gdw&x-1=\pm1 \\
			\gdw&x=0\text{ oder }x=2.
		\end{align*}
	\end{enumerate}
	$\df \nabla f(a)=0\gdw a=\koefff{0}{(-3)\cdot0}{1}=\koefff{0}{0}{1}\text{ oder }a=\koefff{2}{(-3)\cdot2}{1}=\koefff{2}{-6}{1}.$
	
	Es gilt:
	\begin{align*}
		\frac{\partial^2 f}{\partial x^2}(a)&=18x, \\
		\frac{\partial^2 f}{\partial y^2}(a)&=2, \\
		\frac{\partial^2 f}{\partial x\partial y}(a)&=6=\frac{\partial^2 f}{\partial y\partial x}(a). \\
		\df H_f(a)&=\begin{pmatrix}18x & 6 \\ 6 & 2\end{pmatrix}.
	\end{align*}
	\begin{enumerate}
		\item[\underline{$a=\koefff{0}{0}{1}$:}] Es gilt:
		\begin{align*}
			&\determinante(18\cdot0)=0, \\
			&\determinante(\begin{pmatrix}18\cdot0 & 6 \\ 6 & 2\end{pmatrix})=0\cdot2-36=-36. \\
			\df &H_f(\koefff{0}{0}{1})\text{ negativ semidefinit}.
		\end{align*}
		
		\item[\underline{$a=\koefff{2}{-6}{1}$:}] Es gilt:
		\begin{align*}
			&\determinante(18\cdot2)=36, \\
			&\determinante(\begin{pmatrix}18\cdot2 & 6 \\ 6 & 2\end{pmatrix})=36\cdot2-36=36. \\
			\df &H_f(\koefff{2}{-6}{1})\text{ positiv definit}.
		\end{align*}
	\end{enumerate}
	Insgesamt ergibt sich, dass $\koefff{0}{0}{1}$ kein Extremum von $f$ ist. Das einzige lokale Minimum von $f$ ist an der Stelle $\koefff{2}{-6}{1}$.
	\QED  
	
\end{enumerate}

\end{document}