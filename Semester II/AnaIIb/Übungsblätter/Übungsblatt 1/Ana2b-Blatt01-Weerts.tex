\documentclass[12pt]{article}

\usepackage{a4}  
\usepackage{color}
\usepackage{amssymb}
\usepackage{amsmath}
\usepackage[utf8]{inputenc}
\usepackage{faktor}
 
\newcommand{\QED}{\begin{flushright} $\square$ \end{flushright}}
\newcommand{\df}{\enspace\Longrightarrow\enspace}
\newcommand{\koeff}[2]{\begin{pmatrix}#1 \\ #2\end{pmatrix}}
\newcommand{\Char}{\operatorname{char}}
\newcommand{\isIdeal}{\trianglelefteq}
\newcommand{\ann}{\operatorname{ann}}
\newcommand{\ideal}[1]{\langle#1\rangle}
\newcommand{\gdw}{\Leftrightarrow}
\newcommand{\klamm}[4]{\left\{\begin{array}{cc} 
                #1,&\text{falls }#2, \\
                #3,&\text{falls }#4. \\
                \end{array} \right.}



\begin{document}
\section*{Abgabe Analysis IIa, Blatt 01}

Studierende(r): Weerts, Steffen, steffen.weerts@uni-oldenburg.de

\subsection*{Aufgabe 1}
 Sei $M\neq\emptyset$ und $d:M\times M\rightarrow\mathbb{R}$,
	$$d(x,y)=\klamm{1}{x\neq y}{0}{x=y}$$
\begin{enumerate}
	\item[(i)] Zu zeigen: $d$ ist Metrik.
	\begin{enumerate}
		\item[(M1)] Zu zeigen: $d(x,y)\geq0$ und $d(x,y)=0\gdw x=y$. \\
		Es gilt:
		\begin{align*}
			&\forall x,y\in M:d(x,y)=0\text{ oder }d(x,y)=1 \\
			\df &\forall x,y\in M:d(x,y)\geq0.
		\end{align*}
		
		\item[(M2)] Zu zeigen: $d(x,y)=d(y,x)$. \\
		Es gilt:
		\begin{align*}
			d(x,y)=&\klamm{1}{x\neq y}{0}{x=y} \\
			=&\klamm{1}{y\neq x}{0}{y=x} \\
			=&\,d(y,x).
		\end{align*}
		
		\item[(M3)] Zu zeigen: $d(x,z)\leq d(x,y)+d(y,z)$. \\
		Sei oBdA $x\neq z$ (ansonsten gilt $d(x,z)=0$ und nach (M1) gilt $d(x,y)+d(y,z)\geq0$). Es gilt:
		$$d(x,z)=1\leq \begin{array}{cc}1+d(y,z)\\d(x,y)+1\\\end{array}=d(x,y)+d(y,z).$$
	\end{enumerate}
	Da (M1), (M2) und (M3) erfüllt sind, ist $d$ eine Metrik.
	\QED
	
	\item[(ii)] Sei $x_0\in M$. Es gilt:
	$$B_{\frac{1}{3}}={x_0},\text{ da }\forall x\in M,x\neq x_0:d(x_0,x)=1.$$
	$$B_{13}=M,\text{ da }\forall x\in M:d(x_0,x)\leq 1.$$
	\QED
	
	\item[(iii)] Damit eine Folge in $(M,d)$ konvergent gegen $a$ ist, muss $d(a,x_k)\overset{k\rightarrow\infty}{\longrightarrow}0$, d. h. $x_k\overset{k\rightarrow\infty}{\longrightarrow}a$. Da $d(x,y)=1\forall x,y\in M, x\neq y,$ müssen alle Folgenglieder ab einem bestimmten Index $N$ gleich dem Mittelpunkt $a$ sein, d. h.: \\
	\begin{align*}
		&\exists N\in\mathbb{N}\forall n\geq N: x_n=a \\
		\df &\exists N\in\mathbb{N}\forall n\geq N:d(a,x_n)=0 \\
		\df &\forall\varepsilon\geq0\exists N\in\mathbb{N}\forall n\geq N:d(x_n,a)\leq\varepsilon.
	\end{align*}
	\QED
	
	\item[(iv)] Sei $U\subset X, i\in U$ beliebig. \\
	Sei $\varepsilon=\frac{1}{2}$. Es gilt:
	\begin{align*}
		B_{\varepsilon}(x)=&\{a\in X:d(x,a)<\varepsilon\} \\
		=&\{a\in X:d(x,a)<\frac{1}{2}\} \\
		=&\{x\},\quad\quad\text{(da }d(x,a)<1\Leftrightarrow x=a\text) \\
		\subset&\,U.
	\end{align*}
	Daraus folgt: $$\forall x\in X\exists\,\varepsilon>0:B_{\varepsilon}\subset U.$$
	\QED
\end{enumerate}

\subsection*{Aufgabe 2}
\begin{enumerate}
	\item[(a)] Fehlt.
	
	\item[(b)] Fehlt.
\end{enumerate}

\subsection*{Aufgabe 3}
\begin{enumerate}
	\item[(a)] Fehlt.
	
	\item[(b)] Fehlt.
\end{enumerate}

\subsection*{Aufgabe 4}
Fehlt.

\end{document}