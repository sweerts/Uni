\documentclass[oneside]{article}

\usepackage[german]{babel}
\usepackage[latin1]{inputenc}

%\usepackage{fouriernc}
 
%\usepackage{showkeys}
 


\hfuzz 2pt

\usepackage{amsmath,amsfonts}
\usepackage{amsthm}

%\usepackage[mathcal]{euler}


\usepackage{url}


\newcommand{\dint}{\displaystyle\int}


\newcommand{\la}{\langle}


\newcommand{\ra}{\rangle}


\renewcommand{\Tilde}{\widetilde}
\renewcommand{\Hat}{\widehat}
\renewcommand{\Bar}{\overline}

\newcommand{\loc}{\mathrm{loc}}

\newcommand{\TT}{\mathbb{T}}
\newcommand{\RR}{\mathbb{R}}
\newcommand{\ZZ}{\mathbb{Z}}
\newcommand{\CC}{\mathbb{C}}
\newcommand{\NN}{\mathbb{N}}

%\newcommand{\la}{\langle}
%\newcommand{\ra}{\rangle}

\newcommand{\cO}{\mathcal{O}}
\newcommand{\cN}{\mathcal{N}}
\newcommand{\cD}{\mathcal{D}}
\newcommand{\cH}{\mathcal{H}}
\newcommand{\cG}{\mathcal{G}}
\newcommand{\cT}{\mathcal{T}}
\newcommand{\cB}{\mathcal{B}}
\newcommand{\cL}{\mathcal{L}}
\newcommand{\cK}{\mathcal{K}}
\newcommand{\cJ}{\mathcal{J}}
\newcommand{\cV}{\mathcal{V}}
\newcommand{\cE}{\mathcal{E}}
\newcommand{\cU}{\mathcal{U}}
\newcommand{\cS}{\mathcal{S}}
\newcommand{\cA}{\mathcal{A}}
\newcommand{\cF}{\mathcal{F}}

\newcommand{\ELL}{\mathcal{L}}

\DeclareMathOperator{\grad}{Grad}

\renewcommand{\Re}{\mathop{\mathrm{Re}}}
\renewcommand{\Im}{\mathop{\mathrm{Im}}}


%\theoremstyle{theorem}

\renewcommand{\proofname}{\bf Beweis}

\newtheoremstyle{mytheorem}
  {\topsep}   % ABOVESPACE
  {\topsep}   % BELOWSPACE
  {\itshape}  % BODYFONT
  {}       % INDENT (empty value is the same as 0pt)
  {\bfseries} % HEADFONT
  {.}         % HEADPUNCT
  {5pt plus 1pt minus 1pt} % HEADSPACE
  {\thmname{#1}\thmnumber{ #2}\thmnote{ (#3)}}          % CUSTOM-HEAD-SPEC

%\theoremstyle{plain}

\theoremstyle{mytheorem}

\newtheorem{satz}{Satz}
\newtheorem*{satz*}{Satz}
\newtheorem{prop}[satz]{Proposition}
\newtheorem{lemma}[satz]{Lemma}
\newtheorem{corol}[satz]{Korollar}

%\theoremstyle{remark}


\newtheoremstyle{mydefinition}
  {\topsep}   % ABOVESPACE
  {\topsep}   % BELOWSPACE
  {\normalfont}  % BODYFONT
  {}       % INDENT (empty value is the same as 0pt)
  {\bfseries} % HEADFONT
  {.}         % HEADPUNCT
  {5pt plus 1pt minus 1pt} % HEADSPACE
  {\thmname{#1}\thmnumber{ #2}\thmnote{ (#3)}}          % CUSTOM-HEAD-SPEC



\theoremstyle{mydefinition}

\newtheorem{defin}[satz]{Definition}
\newtheorem{bsp}[satz]{Beispiel}
\newtheorem{bmk}[satz]{Bemerkung}
\newtheorem{ubg}{�bung}

%\numberwithin{theorem}{sectio}
%\numberwithin{exer}{section}
%\numberwithin{equation}{section}

\newcommand{\slim}{\mathop{\mathrm{s-lim}}}

\newcommand{\eps}{\varepsilon}

\DeclareMathOperator{\ess}{ess}
\DeclareMathOperator{\supp}{supp}
\DeclareMathOperator{\dist}{dist}
\DeclareMathOperator{\spec}{spec}
%\DeclareMathOperator{\dom}{dom}
\DeclareMathOperator{\res}{res}
\DeclareMathOperator{\ran}{ran}
\DeclareMathOperator{\Ran}{ran}
\DeclareMathOperator{\Id}{Id}
\DeclareMathOperator{\gr}{gr}
\DeclareMathOperator{\vol}{vol}
\DeclareMathOperator{\area}{area}

\newcommand{\specd}{\spec_\mathrm{disc}}
\newcommand{\spece}{\spec_\mathrm{ess}}
\newcommand{\specp}{\spec_\mathrm{p}}

\renewcommand{\proofname}{{\bf Proof}}

%\parindent 0pt
%\parskip 3pt
%
\sloppy

\textwidth 150mm
\textheight 230mm
\oddsidemargin 4.6mm
\topmargin -5.4mm

\begin{document}


\pagestyle{empty}

\begin{center}

\Large\bf

SS 2020 $\bullet$ Analysis IIa $\bullet$ �bungsaufgaben

\bigskip

Blatt 1

\bigskip

{\normalsize\normalfont {\bf Abgabefrist:} bis zum 30.04.2020 um 10 Uhr (als PDF-Datei an den zust�ndigen Tutor)}

\end{center}

\bigskip

\bigskip


{\large

{\bf Aufgabe 1} (2+2+2 Punkte)

\medskip

 Berechnen Sie folgende unbestimmte Integrale:
\[
\text{(a) }
\int (x-1)^2 e^{-x}\, dx,
 \quad
\text{(b) } \int \sqrt{x^2-9}\, dx,
 \quad
\text{(c) }\int (x+1)^2 \cos (2x) \, dx.
\]


\bigskip

\bigskip


{\bf Aufgabe 2} (2+1+2 Punkte) 

\medskip

Berechnen Sie folgende unbestimmte Integrale:
\[
\text{(a) }
\int \sqrt{x} \sin \sqrt x\, dx, \quad \text{(b) }\int \dfrac{\sin \ln x}{x}\, dx,
 \quad \text{(c) }\dint \dfrac{dx}{e^x + 4e^{-x}}.
\]

\bigskip

\bigskip


{\bf Aufgabe 3} (5 Punkte)

\medskip

Berechnen Sie $\dint \dfrac{x}{x^3+8}\, dx$.


\bigskip

\bigskip

{\bf Aufgabe 4} (4 Punkte)

\medskip

Berechnen Sie $\dint \dfrac{dx}{2+\sin x}$.
}%\large

\newpage

\begin{center}

\large\bf Pr�senzaufgaben

\end{center}

\bigskip

\noindent{\bf A.} Sei $a>0$.  Berechnen Sie die Ableitungen folgender Funktionen:
\[
x\mapsto \arctan\dfrac{x}{a}, \quad x\mapsto \arcsin \dfrac{x}{a}, \quad x\mapsto \ln \Big|\dfrac{x-a}{x+a}\Big|,
\quad
x\mapsto \ln \big(x+\sqrt{x^2\pm a^2}\big).
\]
und leiten Sie folgende unbestimmte Integrale her:
\[
\int \dfrac{dx}{\sqrt{x^2\pm a^2}}, \quad \int \dfrac{dx}{\sqrt{a^2-x^2}}, \quad \int \dfrac{dx}{x^2\pm a^2}.
\]

\bigskip

\noindent{\bf B.} Berechnen Sie folgende unbestimmte Integrale:

\begin{tabular}{ll}
\begin{minipage}{60mm}\begin{enumerate}
\item $\dint (x+1)(x-2)\, dx$,
\item $\dint \dfrac{dx}{x^2+4x+5}$,
\item $\dint x^2\cos x\, dx$,
\item $\dint x^2 e^{-3x}\, dx$,
\item $\dint x^3 \ln x\, dx$,
\item $\dint x \arctan x\, dx$,
\item $\dint \sqrt{1-x^2}\, dx$,
\item $\dint \sqrt{x^2+4}\, dx$,
\item $\dint e^x \sin x\, dx$,
\end{enumerate}
\end{minipage}
&
\begin{minipage}{60mm}
\begin{enumerate}
\setcounter{enumi}{9}
\item $\dint e^{\sqrt x}\, dx$,
\item $\dint \sin\sqrt x\, dx$,
\item $\dint \dfrac{1}{(x+1)(x-2)}\, dx$,
\item $\dint \dfrac{1}{x^3-1}$,
\item $\dint \dfrac{x}{x^4-1}$,
\item $\dint \dfrac{dx}{(e^x+1)^2}$,
\item $\dint \sin^2 x\, dx$,
\item $\dint \cos^4 (2x)\, dx$,
\item $\dint x\sin^2 x \, dx$.
\end{enumerate}
\end{minipage}
\end{tabular}

\bigskip

\noindent {\bf C.} Seien $P,Q$ Polynome von zwei Variablen und $R(x,y)=P(x,y)/Q(x,y)$. Wir m�chten zeigen, dass
$\dint R(\cos x,\sin x)\, dx$ immer eine elementare Funktion ist. Daf�r werden wir die Substitution $x=2 \arctan t$ nutzen.

\begin{enumerate}
\item Beweisen Sie die Identit�ten $\cos x= \dfrac{1-t^2}{1+t^2}$ und $\sin x=\dfrac{2t}{1+t^2}$.
\item Leiten Sie her, dass es eine rationale Funktion $t\mapsto r(t)$ existiert mit
\[
\int R(\cos x,\sin x)\, dx = \int r(t)\, dt |_{t=\tan\frac{x}{2}}.
\]
\item Leiten Sie her, dass $\dint R(\cos x,\sin x)\, dx$ eine elementare Funktion ist.

\item Berechnen Sie $\dint \dfrac{dx}{\cos x+\sin x}$.
\end{enumerate}


\end{document}
