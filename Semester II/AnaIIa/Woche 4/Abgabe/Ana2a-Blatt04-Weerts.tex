\documentclass[12pt]{article}

\usepackage{a4}  
\usepackage{color}
\usepackage{amssymb}
\usepackage{amsmath}
\usepackage[utf8]{inputenc}
\usepackage{faktor}
 
\newcommand{\corr}[1]{\textcolor{red}{#1}}
\newcommand{\QED}{\begin{flushright} $\square$ \end{flushright}}
\newcommand{\df}{\enspace\Longrightarrow\enspace}
\newcommand{\koeff}[2]{\begin{pmatrix}#1 \\ #2\end{pmatrix}}
\newcommand{\Char}{\operatorname{char}}
\newcommand{\isIdeal}{\trianglelefteq}
\newcommand{\ann}{\operatorname{ann}}
\newcommand{\ideal}[1]{\langle#1\rangle}
\newcommand{\N}{\operatorname{N}}
\newcommand{\enorm}{\operatorname{d}}
\newcommand{\gdw}{\;\Longleftrightarrow\;}
\newcommand{\abs}[1]{\vert #1\vert}
\newcommand{\ggT}{\operatorname{ggT}}
\newcommand{\kgV}{\operatorname{kgV}}

\newcommand{\aal}{a_{\alpha}}
\newcommand{\ab}{a_{\beta}}
\newcommand{\ba}{b_{\alpha}}
\newcommand{\bb}{b_{\beta}}



\begin{document}
\section*{Abgabe Analysis IIa, Blatt 04}

Studierende(r): Weerts, Steffen, steffen.weerts@uni-oldenburg.de

\subsection*{Aufgabe 1}
\begin{enumerate}
	\item[1.] Fehlt.
	\item[2.] Fehlt.
	\item[3.] Fehlt.
\end{enumerate}

\subsection*{Aufgabe 2}
Fehlt.

\subsection*{Aufgabe 3}
\begin{enumerate}
	\item[(a)] Untersuchen Sie $\int_{0}^{\infty}\frac{dx}{\sqrt{x}+e^x-1}$ auf Konvergenz. \\
	Beh.: $\int_{0}^{\infty}\frac{dx}{\sqrt{x}+e^x-1}$ konvergiert absolut. \\
	Es gilt:
	$$\int_{0}^{\infty}\frac{dx}{\sqrt{x}+e^x-1} = \int_{0}^{1}\frac{dx}{\sqrt{x}+e^x-1} + \int_{1}^{\infty}\frac{dx}{\sqrt{x}+e^x-1}.$$
	Betrachte zunächst $\int_{0}^{1}\frac{dx}{\sqrt{x}+e^x-1}$. Es gilt für alle $x\in(0,1)$:
	$$\left\vert\frac{1}{\sqrt{x}+e^x-1}\right\vert = \frac{1}{\sqrt{x}+e^x-1} \overset{e^x-1\geq 0}{\leq} \frac{1}{\sqrt{x}} = \frac{1}{x^{\frac{1}{2}}}.$$
	Daraus folgt nach Proposition 54, dass $\int_{0}^{1}\frac{dx}{\sqrt{x}+e^x-1}$ absolut konvergiert. \\
	Betrachte nun $\int_{1}^{\infty}\frac{dx}{\sqrt{x}+e^x-1}$. Es gilt für alle $x\in(1,\infty)$:
	$$\left\vert\frac{1}{\sqrt{x}+e^x-1}\right\vert = \frac{1}{\sqrt{x}+e^x-1} \overset{\sqrt{x}-1\geq 0}{\leq} \frac{1}{e^x} \leq \frac{1}{x^2}.$$
	Daraus folgt nach Proposition 54, dass $\int_{1}^{\infty}\frac{dx}{\sqrt{x}+e^x-1}$ absolut konvergiert. \\
	Insgesamt ergibt sich, dass $\int_{0}^{\infty}\frac{dx}{\sqrt{x}+e^x-1} = \int_{0}^{1}\frac{dx}{\sqrt{x}+e^x-1} + \int_{1}^{\infty}\frac{dx}{\sqrt{x}+e^x-1}$ absolut konvergent ist.
	\QED
	
	\item[(b)]Untersuchen Sie $\int_{0}^{1}\frac{\ln(x)}{(1-x)^2}dx$ auf Konvergenz. \\
	Beh.: $\int_{0}^{1}\frac{\ln(x)}{(1-x)^2}dx$ divergiert. \\
	Es gilt:
	\begin{align*}
		\int_{0}^{1}\frac{\ln(x)}{(1-x)^2}dx \overset{PI}{=}& \left.\frac{\ln(x)}{1-x}\right\vert_{\frac{1}{2}}^{1}-\int_{\frac{1}{2}}^{1}\frac{1}{x}\cdot\frac{1}{1-x}dx \\
		=& \left.\frac{\ln(x)}{1-x}\right\vert_{\frac{1}{2}}^{1}-\int_{\frac{1}{2}}^{1}\frac{-1}{x^2-x}dx \\
		=& \left.\frac{\ln(x)}{1-x}\right\vert_{\frac{1}{2}}^{1}-\int_{\frac{1}{2}}^{1}\frac{x-1-x}{x\cdot(x-1)}dx \\
		=& \left.\frac{\ln(x)}{1-x}\right\vert_{\frac{1}{2}}^{1}-\int_{\frac{1}{2}}^{1}\left(\frac{x-1}{x\cdot(x-1)}-\frac{x}{x\cdot(x-1)}\right)dx \\
		=& \left.\frac{\ln(x)}{1-x}\right\vert_{\frac{1}{2}}^{1}-\left(\int_{\frac{1}{2}}^{1}\frac{x-1}{x\cdot(x-1)}dx-\int_{\frac{1}{2}}^{1}\frac{x}{x\cdot(x-1)}dx\right) \\
		=& \left.\frac{\ln(x)}{1-x}\right\vert_{\frac{1}{2}}^{1}-\int_{\frac{1}{2}}^{1}\frac{1}{x}dx+\int_{\frac{1}{2}}^{1}\frac{1}{x-1}dx \\
		=& \left.\frac{\ln(x)}{1-x}\right\vert_{\frac{1}{2}}^{1}-\left.\ln(x)\right\vert_{\frac{1}{2}}^{1}+\left.\ln(x-1)\right\vert_{\frac{1}{2}}^{1} \\
		=& \lim_{k\rightarrow\infty}\frac{\ln(1)}{1-k}-\frac{\ln(\frac{1}{2})}{\frac{1}{2}}-\ln(1)+\ln\left(\frac{1}{2}\right)+\lim_{m\rightarrow 0}\ln(m)-\ln\left(\frac{1}{2}\right) \\
		=& \lim_{k\rightarrow\infty}\frac{0}{1-k}-2\ln\left(\frac{1}{2}\right)-0-\ln(2)+\lim_{m\rightarrow 0}\ln(m)+\ln(2) \\
		=& 0+2\ln(2)+\lim_{m\rightarrow 0}\ln(m) \\
		=& 2\ln(2)+\lim_{m\rightarrow 0}\ln(m) \\
	\end{align*}
	Nun gilt:
	$$2\ln(2)+\lim_{m\rightarrow 0}\ln(m) \longrightarrow 2\ln(2)+\infty = \infty.$$
	Insgesamt ergibt sich, dass $\int_{0}^{1}\frac{\ln(x)}{(1-x)^2}dx$ divergiert.
	\QED
\end{enumerate}

\subsection*{Aufgabe 4}
\begin{enumerate}
	\item[(a)] Berechnen Sie $\int_{-\infty}^{\infty}\frac{dx}{e^x+e^{-x}}$. \\
	Beh.: $\int_{-\infty}^{\infty}\frac{dx}{e^x+e^{-x}} = \frac{\pi}{2}$. \\
	Es gilt:
	\begin{align*}
		\int_{-\infty}^{\infty}\frac{dx}{e^x+e^{-x}} =& 2\cdot\int_{0}^{\infty}\frac{dx}{e^x+e^{-x}} \\
		\overset{x=\ln(t)}{=}& 2\cdot\int_{1}^{\infty}\frac{1}{e^{\ln(t)}+e^{-\ln(t)}}\cdot\frac{1}{t}dt \\
		=& 2\cdot\int_{1}^{\infty}\frac{1}{t+\frac{1}{t}}\cdot\frac{1}{t}dt \\
		=& 2\cdot\int_{1}^{\infty}\frac{1}{t^2+1}dt \\
		=& \left.2\cdot\arctan(t)\right\vert_{1}^{\infty} \\
		=& 2\cdot\left(\lim_{k\rightarrow\infty}\arctan(k)-\arctan(1)\right) \\
		=& 2\cdot\left(\frac{\pi}{2}-\frac{\pi}{4}\right) \\
		=& \frac{\pi}{2}.
	\end{align*}
	\QED
	
	\item[(b)] Berechnen Sie $\int_{0}^{1}\frac{dx}{(2-x)\sqrt{1-x}}dx$. \\
	Beh.: $\int_{0}^{1}\frac{dx}{(2-x)\sqrt{1-x}}dx = \frac{\pi}{2}$. \\
	Es gilt:
	\begin{align*}
		\int_{0}^{1}\frac{dx}{(2-x)\sqrt{1-x}}dx \overset{t:=\sqrt{1-x}}{=}& \int_{1}^{0}\frac{-2t}{(2-(-t^2+1))\cdot t}dt \\
		=& 2\cdot\int_{0}^{1}\frac{dt}{t^2+1} \\
		=& 2\cdot\left(\left.\arctan(t)\right\vert_0^1\right) \\
		=& 2\cdot\left(\arctan(1)-\arctan(0)\right) \\
		=& 2\cdot\left(\frac{\pi}{4}-0\right) \\
		=& \frac{\pi}{2}.
	\end{align*}
	\QED
\end{enumerate}



\bigskip

\corr{korrigiert von \hspace{1cm} am }
\end{document}