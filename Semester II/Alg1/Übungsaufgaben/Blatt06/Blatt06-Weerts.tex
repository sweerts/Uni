\documentclass[12pt]{article}

\usepackage{a4}  
\usepackage{color}
\usepackage{amssymb}
\usepackage{amsmath}
\usepackage[utf8]{inputenc}
\usepackage{faktor}
 
\newcommand{\corr}[1]{\textcolor{red}{#1}}
\newcommand{\QED}{\begin{flushright} $\square$ \end{flushright}}
\newcommand{\df}{\enspace\Longrightarrow\enspace}
\newcommand{\koeff}[2]{\begin{pmatrix}#1 \\ #2\end{pmatrix}}
\newcommand{\Char}{\operatorname{char}}
\newcommand{\isIdeal}{\trianglelefteq}
\newcommand{\ann}{\operatorname{ann}}
\newcommand{\ideal}[1]{\langle#1\rangle}
\newcommand{\N}{\operatorname{N}}
\newcommand{\enorm}{\operatorname{d}}
\newcommand{\gdw}{\;\Longleftrightarrow\;}
\newcommand{\abs}[1]{\vert #1\vert}
\newcommand{\ggT}{\operatorname{ggT}}
\newcommand{\kgV}{\operatorname{kgV}}

\newcommand{\aal}{a_{\alpha}}
\newcommand{\ab}{a_{\beta}}
\newcommand{\ba}{b_{\alpha}}
\newcommand{\bb}{b_{\beta}}



\begin{document}
\section*{Abgabe Algebra 1, Blatt 06}

Studierende(r): Weerts, Steffen, steffen.weerts@uni-oldenburg.de

\subsection*{Aufgabe 6.1}
\begin{enumerate}
	\item[(a)] Zu zeigen: $\faktor{\left(\faktor{\mathbb{Z}}{18\mathbb{Z}}\right)}{\left(\faktor{6\mathbb{Z}}{18\mathbb{Z}}\right)} \cong \faktor{\mathbb{Z}}{6\mathbb{Z}}$. \\
	Es gilt:
	\begin{align*}
		&\mathbb{Z}\text{ Ring, $18\mathbb{Z}$, $6\mathbb{Z}$ Ideale in $R$} \\
		\overset{\text{2. Isom.}}{\df} &\varphi: \faktor{\left(\faktor{\mathbb{Z}}{18\mathbb{Z}}\right)}{\left(\faktor{6\mathbb{Z}}{18\mathbb{Z}}\right)} \rightarrow \faktor{\mathbb{Z}}{6\mathbb{Z}} \text{ Ringisomorphismus} \\
		\df &\faktor{\left(\faktor{\mathbb{Z}}{18\mathbb{Z}}\right)}{\left(\faktor{6\mathbb{Z}}{18\mathbb{Z}}\right)} \cong \faktor{\mathbb{Z}}{6\mathbb{Z}}
	\end{align*}
	\QED
	
	\item[(b)] Fehlt.
	
	\item[(c)] Sei $S=\mathbb{Z}[t], I=t\mathbb{R}[t], R=\mathbb{Z}$. \\
	Zu zeigen: $\faktor{(\mathbb{Z}[t]+t\mathbb{R}[t])}{t\mathbb{R}[t]} \cong \mathbb{Z} \cong \faktor{\mathbb{Z}[t]}{t\mathbb{Z}[t]} \cong \faktor{\mathbb{Z}[t]}{(\mathbb{Z}[t] \cap t\mathbb{R}[t])}$.
	\begin{enumerate}
		\item[(1)] Zu zeigen: $\faktor{(\mathbb{Z}[t]+t\mathbb{R}[t])}{t\mathbb{R}[t]} \cong \faktor{\mathbb{Z}[t]}{(\mathbb{Z}[t] \cap t\mathbb{R}[t])}$. \\
		Es gilt:
		$$\faktor{\mathbb{Z}[t]}{(\mathbb{Z}[t] \cap t\mathbb{R}[t])} \overset{\text{2. Isom.}}{\cong} \faktor{(\mathbb{Z}[t]+t\mathbb{R}[t])}{t\mathbb{R}[t]}.$$
		
		\item[(2)] Zu zeigen: $\faktor{\mathbb{Z}[t]}{t\mathbb{Z}[t]} \cong \faktor{\mathbb{Z}[t]}{(\mathbb{Z}[t] \cap t\mathbb{R}[t])}$
		Es gilt:
		\begin{align*}
			\mathbb{Z}[t]\cap t\mathbb{Z}[t] = &\left\lbrace\sum_{i=0}^{\infty}a_it^i \mid (a_i)_{i\in\mathbb{N}_0}\in\mathbb{Z}^{(\mathbb{N}_0)}\right\rbrace \cap \left\lbrace t\sum_{i=0}^{\infty}a_it^i \mid (a_i)_{i\in\mathbb{N}_0}\in\mathbb{R}^{(\mathbb{N}_0)}\right\rbrace \\
			= &\left\lbrace\sum_{i=0}^{\infty}a_it^i \mid (a_i)_{i\in\mathbb{N}_0}\in\mathbb{Z}^{(\mathbb{N}_0)}\right\rbrace \cap \left\lbrace t\sum_{i=0}^{\infty}a_it^i \mid (a_i)_{i\in\mathbb{N}_0}\in\mathbb{Z}^{(\mathbb{N}_0)}\right\rbrace \\
			= &\left\lbrace t\sum_{i=0}^{\infty}a_it^i \mid (a_i)_{i\in\mathbb{N}_0}\in\mathbb{Z}^{(\mathbb{N}_0)}\right\rbrace \\
			= &t\mathbb{Z}[t].
		\end{align*}
		$\df \faktor{\mathbb{Z}[t]}{t\mathbb{Z}[t]} \cong \faktor{\mathbb{Z}[t]}{(\mathbb{Z}[t] \cap t\mathbb{R}[t])}$.
		
		\item[(3)] Zu zeigen: $\mathbb{Z} \cong \faktor{\mathbb{Z}[t]}{t\mathbb{Z}[t]}$. \\
		Es gilt:
		\begin{align*}
			t\mathbb{Z}[t]\text{ Ideal} \df &\exists\varphi:\mathbb{Z}[t]\rightarrow\mathbb{Z}\text{ Ringhomomorphismus mit }\ker(\varphi)=t\mathbb{Z}[t]. \\
			\overset{\text{Homom.-Satz}}{\df} &\;\psi:\faktor{\mathbb{Z}[t]}{t\mathbb{Z}[t]}\rightarrow\text{Im}(\varphi), [a]\mapsto\varphi(a)\text{ Ringisomorphismus} \\
			\overset{\text{Im}(\varphi)=\mathbb{Z}}{\df} &\;\faktor{\mathbb{Z}[t]}{t\mathbb{Z}[t]}\cong\mathbb{Z}.
		\end{align*}
	\end{enumerate}
	\QED
\end{enumerate}

\subsection*{Aufgabe 6.2}
Sei $R$ Hauptidealring, $g\in R$. \\
Zu zeigen: $a+gR=[a]_g\in\left(\faktor{R}{gR}\right)^*\gdw 1\text{ ist ein $\ggT$ von $a$ und }$g. \\
Beweis fehlt.
\begin{enumerate}
	\item[(a)] Sei $K=\mathbb{Q}, f:=t^3-3t^2+2t,g:=t^2-1$. \\
	Es gilt:
	$$\ggT(t^3-3t^2+2t,t^2-1)\overset{\text{EA}}{=}3t-3 \df [f]_g\notin\left(\faktor{R}{gR}\right)^*.$$
	
	
	\item[(b)] Fehlt.
\end{enumerate}

\subsection*{Aufgabe 6.3}
\begin{enumerate}
	\item[(a)] Zu zeigen: $\psi:\mathbb{Z}\times\mathbb{Z}\rightarrow\faktor{\mathbb{Z}}{3\mathbb{Z}},\;(a,b)\mapsto b+3\mathbb{Z}$ Ringepimorphismus. \\
	\begin{enumerate}
		\item[(1)] Zu zeigen: $\forall a,b\in \mathbb{Z}\times\mathbb{Z}: \psi(a+b)=\psi(a)+\psi(b)$.
		Es gilt:
		\begin{align*}
			\psi(a+b) = &\psi((a_1,a_2)+(b_1,b_2)) \\
			= &a_2+b_2+3\mathbb{Z} \\
			= &a_2+b_2+3\mathbb{Z}+3\mathbb{Z} \\
			= &a_2+3\mathbb{Z}+b_2+3\mathbb{Z} \\
			= &\psi(a)+\psi(b).
		\end{align*}
		
		\item[(2)] Zu zeigen: $\forall a,b\in \mathbb{Z}\times\mathbb{Z}: \psi(a\cdot b)=\psi(a)\cdot\psi(b)$. \\
		Es gilt:
		\begin{align*}
			\psi(a\cdot b) = &\psi((a_1,a_2)\cdot(b_1,b_2)) \\
			= &\psi((a_1b_1,a_2b_2)) \\
			= &a_2b_2+3\mathbb{Z} \\
			= &a_2b_2+a_1\cdot 3\mathbb{Z}+b_1\cdot 3\mathbb{Z} + 3\mathbb{Z}\cdot 3\mathbb{Z} \\
			= &(a_2 + 3\mathbb{Z})\cdot(b_2+3\mathbb{Z}) \\
			= &\psi((a_1,a_2))\cdot\psi((b_1,b_2)) \\
			= &\psi(a)\cdot\psi(b).
		\end{align*}
		
		\item[(3)] Zu zeigen: $\psi((1,1))=1+3\mathbb{Z}$. \\
		Es gilt:
		$$\psi((1,1)) = 1+3\mathbb{Z}.$$	
	\end{enumerate}
	$\df \psi$ Ringhomomorphismus.
	
	Weiter ist zu zeigen, dass $\psi$ surjektiv ist, d. h. $\forall b\in \faktor{\mathbb{Z}}{3\mathbb{Z}} \exists a\in\mathbb{Z}\times\mathbb{Z}:\psi(a)=b$. \\
	Es gilt:
	\begin{align*}
		&\psi((a,b)) = b+3\mathbb{Z} \\
		\overset{b\in\mathbb{Z}}{\df} &\psi\text{ surjektiv}.
	\end{align*}
	$\df \psi$ ist Ringepimorphismus.
	\QED
	
	\item[(b)] Fehlt.
	\end{enumerate}

\subsection*{Aufgabe 6.4}
Sei $K$ Körper, $R=K[t]$. \\
Seien $\ideal{5}$ und $\ideal{7}$ Ideale in $R$. Seien $1$ und $5$ aus $R$. \\
Nach dem Chinesischen Restsatz existiert ein $b\in R$, dass die simultanen Kongruenzen
$$b\equiv 1\mod \ideal{5}$$
$$b\equiv 5\mod \ideal{7}$$
erfüllt. Die Lösung ist eindeutig modulo $\ideal{5}\cdot\ideal{7}=\ideal{35}$.

In diesem Fall ist $b=26$ eine Lösung der Kongruenzen, denn es gilt:
$$26=1+4\cdot5 \equiv 1\mod\ideal{5}\quad\text{ und }\quad 26=5+3\cdot 7\equiv 5\mod\ideal{7}.$$
Alle Lösungen sind in diesem Fall also $26+\ideal{35}$.

\bigskip

\corr{korrigiert von \hspace{1cm} am }
\end{document}