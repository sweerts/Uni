% DAS IST DAS KOMMENTARZEICHEN
% AB HIER NICHTS VERAENDERN, SONST GIBT ES TECHNISCHE PROBLEME
\documentclass[12pt]{article}

\usepackage{a4}  
\usepackage{color}
\usepackage{amssymb}
 \usepackage{amsmath}
 
\newcommand{\corr}[1]{\textcolor{red}{#1}}%Korrekturmarkierungen

\begin{document}
% AB HIER VERAENDERUNGEN ERLAUBT:
\textcolor{blue}{Bitte Abgaben als Mail an den Tutor Ihrer \"Ubungsgruppe. 
Der Betreff der Mail sollte sein   \\
\begin{center}
   Abgabe Algebra 1, Blatt XX  
\end{center}
wobei XX durch die Nummer des Blattes (00,01,...,12) ist. Der Dateiname der
angef\"ugten Abgabe sollte sein\\
\begin{center}
   BlattXX-Nachname.tex
\end{center}
wobei XX wieder die Nummer des Blattes und Nachname Ihr Nachname ist.}

\section*{Abgabe Algebra 1, Blatt 00}

Studierende(r): Name, Vorname, UOL-Mailadresse % eigene Daten einfügen

\subsection*{Aufgabe 0.1}  

\subsection*{Aufgabe 0.2}

$$\forall \, n \in \mathbb{N}, \; n\geq 2 \;\exists !\, d \in \mathbb{N}\; \exists !\, p_1 \, ,\, \ldots \, , \,p_d\in \mathbb{P}: n=\prod_{i=1}^{d} p_i$$
% Die \, und \; sind nur für die Leerzeilen, also eigentlich hier nicht notwendig.
$$360 = 2 \cdot 2 \cdot 2 \cdot 3 \cdot 3 \cdot 5$$

\subsection*{Aufgabe 0.3}

$\forall \, a \in \mathbb{Z} \; \forall \, b \in \mathbb{N} \; \exists ! \, q,r \in \mathbb{Z}: a=q \cdot b + r \text{ und } 0 \leq r < b$ 
% In der einfachen $-Umgebung wird das Geschriebene linksbündig ausgegeben. Im Gegensatz zu Text, der zwischen doppelten $-Zeichen steht, wird auch nicht automatisch ein Absatz gemacht. Diesen kann man mit \\ hinzufügen.
\medskip  \\
% \medskip vergrößert den Zeilenabstand an dieser Stelle.
Hier ein eigenes Beispiel mit $a=47$ und $b=6$:
$$a=47=42+5=7 \cdot 6 + 5= 7 \cdot b + 5$$
Hier sind $q=7$ und $r=5$.\\





% HIERUNTER WIEDER NICHTS VERAENDERN, SONST GIBT ES TECHNISCHE PROBLEME

\bigskip

\corr{korrigiert von \hspace{1cm} am }
\end{document}