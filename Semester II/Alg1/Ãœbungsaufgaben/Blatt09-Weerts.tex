\documentclass[12pt]{article}

\usepackage{a4}  
\usepackage{color}
\usepackage{amssymb}
\usepackage{amsmath}
\usepackage[utf8]{inputenc}
\usepackage{faktor}
 
\newcommand{\corr}[1]{\textcolor{red}{#1}}
\newcommand{\QED}{\begin{flushright} $\square$ \end{flushright}}
\newcommand{\QEDD}{\begin{flushright} "$\square$" \end{flushright}}
\newcommand{\df}{\enspace\Longrightarrow\enspace}
\newcommand{\koeff}[2]{\begin{pmatrix}#1 \\ #2\end{pmatrix}}
\newcommand{\koefff}[3]{\begin{pmatrix}#1 \\ #2 \\ #3\end{pmatrix}}
\newcommand{\Char}{\operatorname{char}}
\newcommand{\isIdeal}{\trianglelefteq}
\newcommand{\ann}{\operatorname{ann}}
\newcommand{\ideal}[1]{\langle#1\rangle}
\newcommand{\N}{\operatorname{N}}
\newcommand{\enorm}{\operatorname{d}}
\newcommand{\gdw}{\;\Longleftrightarrow\;}
\newcommand{\abs}[1]{\vert #1\vert}
\newcommand{\ggT}{\operatorname{ggT}}
\newcommand{\kgV}{\operatorname{kgV}}
\newcommand{\grad}{\operatorname{deg}}
\newcommand{\LC}{\operatorname{LC}}
\newcommand{\determinante}{\operatorname{det}}

\newcommand{\aal}{a_{\alpha}}
\newcommand{\ab}{a_{\beta}}
\newcommand{\ba}{b_{\alpha}}
\newcommand{\bb}{b_{\beta}}



\begin{document}
\section*{Abgabe Algebra I, Blatt 09}

Studierende(r): Weerts, Steffen, steffen.weerts@uni-oldenburg.de

\subsection*{Aufgabe 9.1}
\begin{enumerate}
	\item[(a)] Fehlt.
	
	\item[(b)] 
	\begin{enumerate}
		\item[(i)] Sei $f_1=t^4-5\in\mathbb{Q}[t]$. \\
		Es gilt:
		\begin{align*}
			f_1&=t^4-5 \\
			&=(t^2+\sqrt{5})(t^2-\sqrt{5}) \\
			&=(t+\sqrt[4]{5}i)(t-\sqrt[4]{5}i)(t+\sqrt[4]{5})(t-\sqrt[4]{5})
		\end{align*}
		Da $\pm\sqrt[4]{5},\pm\sqrt[4]{5}i\in\mathbb{Q}(\sqrt[4]{5},i)$, zerfällt $f_q$ über $\mathbb{Q}(\sqrt[4]{5},i)$. \\
		
		Zu zeigen: $[\mathbb{Q}(\sqrt[4]{5},i):\mathbb{Q}(\sqrt[4]{5})]=2$. \\
		Da $f_{i,\mathbb{Q}(\sqrt[4]{5})}=t^2+1$, ist $[\mathbb{Q}(\sqrt[4]{5},i):\mathbb{Q}(\sqrt[4]{5})]=\grad(f_{i,\mathbb{Q}(\sqrt[4]{5})})=2$. \\
		
		Zu zeigen: $[\mathbb{Q}(\sqrt[4]{5}):\mathbb{Q}]=4$. \\
		Angenommen, $\grad(f_{\sqrt[4]{5},\mathbb{Q}})=1$. \\
		Es gilt:
		\begin{align*}
			f_{\sqrt[4]{5},\mathbb{Q}}(\sqrt[4]{5})=\sqrt[4]{5}+a_0\overset{!}{=}0 \\
			\df a_0=-\sqrt[4]{5}\notin\mathbb{Q} \\
			\df \grad(f_{\sqrt[4]{5},\mathbb{Q}})\neq1.
		\end{align*}
		Angenommen, $\grad(f_{\sqrt[4]{5},\mathbb{Q}})=2$. \\
		Es gilt:
		\begin{align*}
			&f_{\sqrt[4]{5},\mathbb{Q}}(\sqrt[4]{5})=\sqrt[4]{5}^2+a_1\sqrt[4]{5}+a_0\overset{!}{=}0 \\
			\df &a_0=-\sqrt{5}-a_1\sqrt[4]{5}\notin\mathbb{Q}\quad\forall a_1\in\mathbb{Q} \\
			\df &\grad(f_{\sqrt[4]{5},\mathbb{Q}})\neq2.
		\end{align*}
		Angenommen, $\grad(f_{\sqrt[4]{5},\mathbb{Q}})=3$. \\
		Es gilt:
		\begin{align*}
			&f_{\sqrt[4]{5},\mathbb{Q}}(\sqrt[4]{5})=\sqrt[4]{5}^3+a_2\sqrt[4]{5}^2+a_1\sqrt[4]{5}+a_0 \\
			&\quad\quad\quad\quad=\sqrt{5}\sqrt[4]{5}+a_2\sqrt{5}+a_1\sqrt[4]{5}+a_0\overset{!}{=}0 \\
			\df &a_0=-\sqrt{5}\sqrt[4]{5}-a_2\sqrt{5}-a_1\sqrt[4]{5}\notin\mathbb{Q}\quad\forall a_1,a_2\in\mathbb{Q} \\
			\df &\grad(f_{\sqrt[4]{5},\mathbb{Q}})\neq3.
		\end{align*}
		Angenommen, $\grad(f_{\sqrt[4]{5},\mathbb{Q}})=4$. \\
		Es gilt:
		\begin{align*}
			&f_{\sqrt[4]{5},\mathbb{Q}}(\sqrt[4]{5})=\sqrt[4]{5}^4+a_3\sqrt[4]{5}^3+a_2\sqrt[4]{5}^2+a_1\sqrt[4]{5}+a_0 \\
			&\quad\quad\quad\quad=5+a_3\sqrt{5}\sqrt[4]{5}+a_2\sqrt{5}+a_1\sqrt[4]{5}+a_0\overset{!}{=}0 \\
			\df &a_0=-5-a_3\sqrt{5}\sqrt[4]{5}-a_2\sqrt{5}-a_1\sqrt[4]{5}\in\mathbb{Q}\text{ für }a_1=a_2=a_3=0 \\
			\df &f_{\sqrt[4]{5},\mathbb{Q}}=t^4-5 \\
			\df &[\mathbb{Q}(\sqrt[4]{5}):\mathbb{Q}]=\grad(f_{\sqrt[4]{5},\mathbb{Q}})=4.
		\end{align*}
		Insgesamt ergibt sich: $$[\mathbb{Q}(\sqrt[4]{5},i):\mathbb{Q}]=[\mathbb{Q}(\sqrt[4]{5},i):\mathbb{Q}(\sqrt[4]{5})]\cdot[\mathbb{Q}(\sqrt[4]{5}):\mathbb{Q}]=2\cdot4=8.$$
		\QED
		
		\item[(ii)] Sei $f_2=t^4+1$. \\
		Es gilt:
		\begin{align*}
			f_2&=t^4+1 \\
			&=(t^2+i)(t^2-i) \\
			&=(t+i\sqrt{i})(t-i\sqrt{i})(t+\sqrt{i})(t-\sqrt{i}).
		\end{align*}
		Da $\sqrt{i},i\sqrt{i}\notin\mathbb{Q}$, zerfällt $f_2$ nicht über $\mathbb{Q}$, jedoch über $\mathbb{Q}(i,\sqrt{i})$, denn  $\sqrt{i},i\sqrt{i}$ sind die Nullstellen von $f_2$ in $\mathbb{Q}(i,\sqrt{i})$. \\
		
		Zu zeigen: $[\mathbb{Q}(i):\mathbb{Q}]=2$. \\
		Es gilt:
		\begin{align*}
			&f_{i,\mathbb{Q}}=t^2+1. \\
			\df &[\mathbb{Q}(i):\mathbb{Q}]=\grad(f_{i,\mathbb{Q}})=2.
		\end{align*}
		
		Zu zeigen: $[\mathbb{Q}(i,\sqrt{i}):\mathbb{Q}(i)]=2$. \\
		Angenommen, $\grad(f_{\sqrt{i},\mathbb{Q}(i)})=1$. \\
		Es gilt:
		\begin{align*}
			f_{\sqrt{i},\mathbb{Q}(i)}(\sqrt{i})=\sqrt{i}+a_0\overset{!}{=}0 \\
			\df a_0=-\sqrt{i}\notin\mathbb{Q}(i) \\
			\df \grad(f_{\sqrt{i},\mathbb{Q}(i)})\neq1.
		\end{align*}
		Angenommen, $\grad(f_{\sqrt{i},\mathbb{Q}(i)})=2$. \\
		Es gilt:
		\begin{align*}
			&f_{\sqrt{i},\mathbb{Q}(i)}(\sqrt{i})=\sqrt{i}^2+a_1\sqrt{i}+a_0 \\
			&\quad\quad\quad=i+a_1\sqrt{i}+a_0\overset{!}{=}0 \\
			\df &a_0=-i-a_1\sqrt{i} \\
			\df &(a_0\in\mathbb{Q}(i)\gdw a_1=0) \\
			\df &f_{\sqrt{i},\mathbb{Q}(i)}=t^2-i \\
			\df &[\mathbb{Q}(i,\sqrt{i}):\mathbb{Q}(i)]=\grad(f_{\sqrt{i},\mathbb{Q}(i)})=2.
		\end{align*}
		Insgesamt ergibt sich: $$[\mathbb{Q}(i,\sqrt{i}):\mathbb{Q}]=[\mathbb{Q}(i,\sqrt{i}):\mathbb{Q}(i)]\cdot[\mathbb{Q}(i):\mathbb{Q}]=2\cdot2=4.$$
		\QED
		
	\end{enumerate}
	\item[(c)] 
	\begin{enumerate}
		\item[(i)] Sei $K:=\mathbb{Q}\subseteq\mathbb{Q}(i)=:L$ Körpererweiterung, $f=t^2+1\in K[t],n:=\grad(f)=2$. \\
		Es gilt:
		\begin{align*}
			f&=t^2+1 \\
			&=t^2-i^2 \\
			&=(t+i)(t-i).
		\end{align*}
		Außerdem gilt: $$t+i,t-i\in L[t]\df f\text{ zerfällt über }L\text{, aber nicht über K, da }i\notin K.$$		
		Zu zeigen: $[L:K]=2$. \\
		Da $i\notin\mathbb{Q}$ ist, muss der Grad des Minimalpolymoms von $i$ größer als $1$ sein. Da $i^2+1=0$, ist $f$ das Minimalpolynom von $i$ über $\mathbb{Q}$. \\
		Es gilt: $$[L:K]=\grad(f)=2.$$
		\QED
		
		\item[(ii)] Fehlt.
		\item[(iii)] Fehlt.
	\end{enumerate}
\end{enumerate}

\subsection*{Aufgabe 9.2}
\begin{enumerate}
	\item[(a)] Sei $f:=t^4+t^3+2t^2+1\in\mathbb{Z}_3[t]$. \\
	Zu zeigen: $f=(t+1)\cdot(t^3+2t+1)$ ist eine Zerlegung von $f$ in irreduzible Polynome über $\mathbb{Z}_3$. \\
	Es gilt:
	\begin{align*}
		f&=t^4+t^3+2t^2+1 \\
		&=t^4+t^3+2t^2+2t+t+1 \\
		&=(t+1)(t^3+2t+1).
	\end{align*}
	Zu zeigen: $t+1$ irreduzibel über $\mathbb{Z}_3$. \\
	Es gilt: $$\mathbb{Z}_3\text{ Körper und }\grad(t+1)=1\overset{5.1.2}{\df} t+1\text{ irreduzibel über }\mathbb{Z}_3.$$
	Zu zeigen: $h:=t^3+2t+1$ irreduzibel über $\mathbb{Z}_3$. \\
	Angenommen, $h$ sei reduzibel über $\mathbb{Z}_3$. Da $\mathbb{Z}_3$ Körper ist und $\grad(h)=3$ hat $h$ nach Beobachtung 5.1.6 eine Nullstelle in $\mathbb{Z}_3$. Es gilt:
	\begin{align*}
		h(0)&=0^3+2\cdot0+1=1\neq0, \\
		h(1)&=1^3+2\cdot1+1=1\neq0, \\
		h(2)&=2^3+2\cdot2+1=1\neq0.
	\end{align*}
	Dies steht im Widerspruch zu Beobachtung 5.1.6, weshalb $h$ nicht reduzibel über $\mathbb{Z}_3$ sein kann. \\
	Insgesamt ergibt sich, dass $f=(t+1)\cdot(t^3+2t+1)$ eine Zerlegung von $f$ in irreduzible Polynome über $\mathbb{Z}_3$ ist.
	\QED
	
	\item[(b)] Fehlt.
	\item[(c)] Fehlt.
\end{enumerate}

\subsection*{Aufgabe 9.3}
Sei $R:=\mathbb{Z}_6$ und $M:=R\times R$. \\
Zu zeigen: $X:=((2,4))$ linear unabhängig. \\
Es gilt:
\begin{align*}
	&0\neq3\in R \\
	\df &3\cdot(2,4) = (3\cdot2,3\cdot4)=(0,0) \\
	\df &((2,4))\text{ ist eine linear abhängige Familie}.
\end{align*}
\QED


\bigskip

\corr{korrigiert von \hspace{1cm} am }
\end{document}