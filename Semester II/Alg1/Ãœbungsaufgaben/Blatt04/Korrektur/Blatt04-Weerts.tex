\documentclass[12pt]{article}

\usepackage{a4}  
\usepackage{color}
\usepackage{amssymb}
\usepackage{amsmath}
\usepackage[utf8]{inputenc}
\usepackage{faktor}
 
\newcommand{\corr}[1]{\textcolor{red}{#1}}
\newcommand{\QED}{\begin{flushright} $\square$ \end{flushright}}
\newcommand{\df}{\enspace\Longrightarrow\enspace}
\newcommand{\koeff}[2]{\begin{pmatrix}#1 \\ #2\end{pmatrix}}
\newcommand{\Char}{\operatorname{char}}
\newcommand{\isIdeal}{\trianglelefteq}
\newcommand{\ann}{\operatorname{ann}}
\newcommand{\ideal}[1]{\langle#1\rangle}
\newcommand{\N}{\operatorname{N}}
\newcommand{\enorm}{\operatorname{d}}
\newcommand{\gdw}{\;\Longleftrightarrow\;}
\newcommand{\abs}[1]{\vert #1\vert}
\newcommand{\ggT}{\operatorname{ggT}}
\newcommand{\kgV}{\operatorname{kgV}}

\newcommand{\aal}{a_{\alpha}}
\newcommand{\ab}{a_{\beta}}
\newcommand{\ba}{b_{\alpha}}
\newcommand{\bb}{b_{\beta}}



\begin{document}
\section*{Abgabe Algebra 1, Blatt 04}

Studierende(r): Weerts, Steffen, steffen.weerts@uni-oldenburg.de

\subsection*{Aufgabe 4.1}
\begin{enumerate}
	\item[(a)] Sei $R$ euklidischer Ring. \\
	Zu zeigen: $R$ Hauptidealring. \\
	Sei $I\isIdeal R$ beliebig. Sei $b\in I\setminus{0}$, sodass $\enorm(b)$ das kleinste Element der Menge $M=\{\enorm(x)\mid x\in I\setminus\{0\}\}\subseteq\mathbb{N}_0$ ist. \\
\corr{Was ist mit $I=\{0\}$? $-0,5$ P.}\\
	Beh.: $I=\ideal{b}$.
	\begin{enumerate}
		\item["$\supseteq$":] Es gilt: $$b\in I\df\ideal{b}\subseteq I.$$
		
		\item["$\subseteq$":] Sei $a\in I$ beliebig. \\
		Da $b\neq 0$ und $R$ euklidischer Ring, gilt:
		\begin{align*}
			&\exists q,r\in R: a=qb+r\text{ und }\left(r=0\text{ oder }\enorm(r)<\enorm(b)\right) \\
			\df &r=a-qb\in I,\text{ da $I$ Ideal, $a\in I$, $b\in I$ und $q\in R$}.
		\end{align*}
		Da $\enorm(b)$ das kleinste Element in $M$ ist, gilt: $$r=0\df a=qb\df I=\ideal{b}.$$
		Daraus folgt, dass jedes Ideal in $R$ Hauptideal ist, also ist $R$ Haupt- idealring.
		\QED
\corr{Punkte Teil a): $2,5/3$}
	\end{enumerate}
	
	\item[(b)] Sei $R$ ein euklidischer Ring und seien $a,b\in R$ mit $b\neq 0$. Sei $d\in R$ ein größter gemeinsamer Teiler von a und b. \\
	Zu zeigen: $\ideal{a,b}=\ideal{d}$.\\
	
	Sei $r_{n+1}$ aus dem euklidischen Algorithmus der Rest, der gleich $0$ ist. \\
	Aufgrund der induktiven Vorgehensweise des euklidischen Algorithmus lässt sich jedes $r_i, i\in\{1,\cdots,n\}$ als Linearkombination von $a$ und $b$ schreiben, denn es gilt:
	\begin{align*}
		r_{-1}=q_{0}r_{0}+r_{1} &\gdw r_{1}=r_{-1}-q_{0}r_{0} \\
		r_{0}=q_{1}r_{1}+r_{2} &\gdw r_{2}=r_{0}-q_{1}r_{1} \\
		r_{1}=q_{2}r_{2}+r_{3} &\gdw r_{3}=r_{1}-q_{2}r_{2} \\
		&\;\;\,\cdots  \\
		r_{n-2}=q_{n-1}r_{n-1}+r_{n} &\gdw r_{n}=r_{n-2}-q_{n-1}r_{n-1}
	\end{align*}
	Durch Einsetzen der Gleichungen $r_i=r_{i-2}-q_{i-1}r_{i-1}\forall i\in\{1,\cdots,n-1\}$ erhält man eine Lösung für $r_n$, die eine Linearkombination von $r_{-1}=a$ und $r_0=b$ ist. \\
	Da $r_n=:d$ der größte gemeinsame Teiler von $a$ und $b$ ist, sind alle Linearkombinationen von $a$ und $b$ Vielfache von $d$. Daraus folgt $\ideal{a,b}=\ideal{d}$.\\
\corr{$r_n$ ist bereits aus dem erweiterten euklidischen Algorithmus definiert, kann also nicht mehr als $d$ gesetzt werden. $-0,5$ P}
	\QED
\corr{Punkte Teil b): $2,5/3$}
	
\end{enumerate}
\corr{$5/6$ P}


\subsection*{Aufgabe 4.2}
\begin{enumerate}
	\item[(a)] Fehlt.
	
	\item[(b)] Sei $R=\mathbb{Z}$, $a,b\in\mathbb{Z},b\neq 0$. \\
	Zu zeigen: $\ggT(a,b)\cdot\kgV(a,b)=\abs{ab}$. \\
	Es gilt:
	\begin{align}
		&\kgV(a,b)\mid ab \nonumber\\
		\df &\exists c\in\mathbb{Z}:\kgV(a,b)\cdot c=ab \nonumber \\
		\df &c=\frac{ab}{\kgV(a,b)}
	\end{align}
	Außerdem gilt:
	\begin{align*}
		&c\cdot \frac{\kgV(a,b)}{b}=a\text{ und }c\cdot \frac{\kgV(a,b)}{a}=b \\
		\df &c\mid a\text{ und }c\mid b \\
		\df &c\text{ ist gemeinsamer Teiler von $a$ und $b$}.
	\end{align*}
	Sei $d$ eine weiterer gemeinsamer Teiler von $a$ und $b$. \\
	Zu zeigen: $d\mid c$. \\
	Es gilt:
	\begin{align*}
		&\frac{ab}{d}=a\cdot\frac{b}{d}=\frac{a}{d}\cdot b \\
		\df &a\mid\frac{ab}{d}\text{ und }b\mid\frac{ab}{d} \\
		\df &\kgV(a,b)\mid\frac{ab}{d} \\
		\df &\exists z\in\mathbb{Z}:\kgV(a,b)\cdot z=\frac{ab}{d} \\
		&\gdw d\cdot\kgV(a,b)\cdot z=ab \\
		&\overset{(1)}{\gdw} d\cdot\kgV(a,b)\cdot z=c\cdot\kgV(a,b) \\
		&\gdw d\cdot z=c \\
		&\gdw d\mid c \\
		\df &c=\ggT(a,b).
	\end{align*}
\corr{$c$ ist ein größter gemeinsamer Teiler, allerdings nicht unbedingt der ggT, da dieser in $\mathbb{Z}$ als positiv definiert ist, $c$ allerdings negativ sein kann. $-1$ P}\\
	Insgesamt ergibt sich also: $$\ggT(a,b)\cdot\kgV(a,b)=ab.$$
	\QED
\corr{Punkte Teil b): $2/3$}
\end{enumerate}
\corr{$2/6$ P}



\subsection*{Aufgabe 4.3}
\begin{enumerate}
	\item[(a)] 
	\begin{enumerate}
		\item[(i)] Sei $\mathbb{Z}[i\sqrt{n}]=\left\lbrack a+ib\sqrt{n}\mid a,b\in\mathbb{Z}\right\rbrack$ Unterring von $\mathbb{C}$. \\
		Sei $$N:\mathbb{Z}[i\sqrt{n}]\rightarrow\mathbb{N}_0, \quad
		\alpha:=a+ib\sqrt{n}\mapsto \N(\alpha)=\alpha\bar{\alpha}=a^2+nb^2.$$
		Zu zeigen: $\forall\alpha,\beta\in\mathbb{Z}[i\sqrt{n}]:\N(\alpha\cdot\beta)=\N(\alpha)\N(\beta).$ \\
		Seien $\alpha,\beta\in\mathbb{Z}[i\sqrt{n}]$ beliebig. Es gilt:
		\begin{align*}
			\N(\alpha\beta)=&(\aal+i\ba\sqrt{n})(\ab+i\bb\sqrt{n})\overline{(\aal+i\ab\sqrt{n})(\ab+i\bb\sqrt{n})} \\
			=&(\aal+i\ba\sqrt{n})(\ab+i\bb\sqrt{n})(\aal-i\ab\sqrt{n})(\ab-i\bb\sqrt{n}) \corr{*} \\
			=&(\aal+i\ba\sqrt{n})(\ab+i\bb\sqrt{n})(\aal\ab-i\ba\sqrt{n}\ab-\aal i\bb\sqrt{n}-\ba\sqrt{n}\bb\sqrt{n}) \\
			=&(\aal+i\ba\sqrt{n})(\ab\aal\ab-\ab i\ba\sqrt{n}\ab-\ab\aal i\bb\sqrt{n}-\ab\ba\sqrt{n}\bb\sqrt{n} \\
			&+i\bb\sqrt{n}\aal\ab-i\bb\sqrt{n}i\ba\sqrt{n}\ab-i\bb\sqrt{n}\aal i\bb\sqrt{n}-i\bb\sqrt{n}\ba\sqrt{n}\bb\sqrt{n}) \\
			=&\aal\ab\aal\ab-\aal\ab i\ba\sqrt{n}\ab-\aal\ab\aal i\bb\sqrt{n}-\aal\ab\ba\sqrt{n}\bb\sqrt{n} \\
			&+\aal i\bb\sqrt{n}\aal\ab-\aal i\bb\sqrt{n}i\ba\sqrt{n}\ab-\aal i\bb\sqrt{n}\aal i\bb\sqrt{n}-\aal i\bb\sqrt{n}\ba\sqrt{n}\bb\sqrt{n} \\		
			&+i\ba\sqrt{n}\ab\aal\ab-i\ba\sqrt{n}\ab i\ba\sqrt{n}\ab-i\ba\sqrt{n}\ab\aal i\bb\sqrt{n}-i\ba\sqrt{n}\ab\ba\sqrt{n}\bb\sqrt{n} \\
			&+i\ba\sqrt{n}i\bb\sqrt{n}\aal\ab-i\ba\sqrt{n}i\bb\sqrt{n}i\ba\sqrt{n}\ab-i\ba\sqrt{n}i\bb\sqrt{n}\aal i\bb\sqrt{n} \\
			&-i\ba\sqrt{n}i\bb\sqrt{n}\ba\sqrt{n}\bb\sqrt{n} \\
			=&\aal^2\ab^2+\aal^2\bb^2n+\ab^2\ba^2n+\ba^2\bb^2n^2 \\
			=&(\aal^2+n\ba^2)(\ab^2+n\bb^2) \\
			=&\N(\alpha)\N(\beta).
		\end{align*}
\corr{Du hast bei * verwendet, dass $\overline{\alpha \beta}=\bar{\alpha}\bar{\beta}$. Das musste noch gezeigt werden. $-0,5$ P}
		\QED
		
		\item[(ii)] Sei $\mathbb{Z}[i\sqrt{n}]=\left\lbrack a+ib\sqrt{n}\mid a,b\in\mathbb{Z}\right\rbrack$ Unterring von $\mathbb{C}$. \\
		Sei $$N:\mathbb{Z}[i\sqrt{n}]\rightarrow\mathbb{N}_0, \quad
		\alpha:=a+ib\sqrt{n}\mapsto \N(\alpha)=\alpha\bar{\alpha}=a^2+nb^2.$$
		Zu zeigen: $\N(\alpha)=1\gdw\alpha\in\mathbb{Z}[i\sqrt{n}]^*$.
		\begin{enumerate}
			\item["$\Rightarrow$"] Sei $\N(a)=a^2+nb^2=1$. \\
			Es gilt:
			\begin{align*}
				&a^2+nb^2=1 \\
				\df &(a=0, n=1, b=\pm 1)\text{ oder }(b=0, a=1) \\
&\corr{\text{Warum gilt das? Genau begründen. $-0,5$ P}}\\
				\df &\alpha=1\text{ oder }(\alpha=\pm i,n=1) \\
&\corr{\text{Was ist mit $\alpha=-1$? $-0,5$ P}}\\
				\df &\mathbb{Z}[i\sqrt{n}]\ni\alpha=1=\frac{1}{1}\text{ oder }\left(\mathbb{Z}[i\sqrt{1}]\ni\alpha=\pm i\df\frac{i}{i}=1\in\mathbb{Z}[i\sqrt{1}]\right) \\
				\df &\alpha\in\mathbb{Z}[i\sqrt{n}]^*.
			\end{align*}
\corr{Warum ist $\alpha$ eine Einheit, wenn $\alpha \in \{\pm1,\pm i\}$? Genau begründen.}
			
			\item["$\Leftarrow$"] Sei $\alpha\in\mathbb{Z}[i\sqrt{n}]^*$. \\
			Es gilt:
			\begin{align*}
				&\alpha\in\mathbb{Z}[i\sqrt{n}]^* \\
				\df &\exists\beta\in\mathbb{Z}[i\sqrt{n}]^*:\alpha\beta=1 \\
				\df &\alpha\beta=(\aal+i\ba\sqrt{n})(\ab+i\bb\sqrt{n}) \\
				&=\aal\ab+\aal*i\bb\sqrt{n}+\ab*i\ba\sqrt{n}+i^2\ba\bb*n \\
				&=\aal\ab+\aal*i\bb\sqrt{n}+\ab*i\ba\sqrt{n}-\ba\bb*n \\
				&=1 \\
				\df &\aal*i\bb\sqrt{n}+\ab*i\ba\sqrt{n}=0\text{ und }\aal\ab-\ba\bb*n=1
			\end{align*}
			Sei $\beta\neq\bar{\alpha}$.
			Es gilt:
			\begin{align*}
				\alpha\beta=&1 \\
				\neq&\aal\ab+\aal*i\bb\sqrt{n}+\ab*i\ba\sqrt{n}-\ba\bb*n \\
				=&\aal^2+\aal*i(-\ba)\sqrt{n}+\aal*i\ba\sqrt{n}-\ba(-\ba)*n \\
				=&\aal^2+\ba^2*n \\
				=&1.\text{ Widerspruch $\alpha\beta=1\neq 1$.} \\
				\df &\beta=\bar{\alpha} \\
				\df &\N(\alpha)=\alpha\bar{\alpha}=1.
			\end{align*}
\corr{Nich nachvollziehbar. Warum ist $\aal^2+\ba^2*n =1$? Das war doch zu zeigen. $-1$ P}
		\end{enumerate}
		Wenn $n\geq 2$, dann ist $\mathbb{Z}[i\sqrt{n}]^*=\{\pm 1\}$, denn es gilt:
		\begin{align*}
			&\N(\alpha)=1\gdw \alpha\in\mathbb{Z}[i\sqrt{n}]^* \\
			\df &\N(\alpha)=a^2+n*b^2\geq a^2+2*b^2
		\end{align*}
		Nun gilt für $n=2$:
		\begin{align*}
			\abs{b}=0 &\df\N(\alpha)=a^2. \\
			\abs{b}=1 &\df\N(\alpha)=a^2+2. \\
			\abs{b}\geq 2 &\df\N(\alpha)\geq a^2+4.
		\end{align*}
		Außerdem gilt für $n>2$:
		\begin{align*}
			\abs{b}=0 &\df\N(\alpha)=a^2. \\
			\abs{b}=1 &\df\N(\alpha)>a^2+2. \\
			\abs{b}\geq 2 &\df\N(\alpha)>a^2+4.
		\end{align*}
		Daraus folgt, dass $\N(\alpha)=1\gdw b=0$ für $n\geq 2$ gilt.
		\QED
\corr{Es ist noch zu zeigen, dass $\alpha=\pm1$ ist. $-0,5$ P}\\
\corr{Punkte Teil a): $1/4$}
	\end{enumerate}
		
	\item[(b)] Fehlt.
\end{enumerate}
\corr{$1/8$ P}

\bigskip

\corr{Insgesamt $8/20$ Punkten.}


\bigskip

\corr{korrigiert von Tom Engels am 21.05.2020}
\end{document}