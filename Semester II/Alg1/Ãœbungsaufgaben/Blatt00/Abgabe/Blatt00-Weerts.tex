% DAS IST DAS KOMMENTARZEICHEN
% AB HIER NICHTS VERAENDERN, SONST GIBT ES TECHNISCHE PROBLEME
\documentclass[12pt]{article}

\usepackage{a4}  
\usepackage{color}
\usepackage{amssymb}
 \usepackage{amsmath}
 
\newcommand{\corr}[1]{\textcolor{red}{#1}}%Korrekturmarkierungen

\begin{document}

\section*{Abgabe Algebra 1, Blatt 00}

Studierende(r): Weerts, Steffen, steffen.weerts@uni-oldenburg.de % eigene Daten einfügen

\subsection*{Aufgabe 0.1}
Hier steht $$x_1^3$$

\subsection*{Aufgabe 0.2}
$$\forall n \in \mathbb{N} \; \exists! M \in \mathbb{N} \; \exists! p_1, \ldots, p_M \in \mathbb{P}: n=\prod_{k=1}^{M}p_k$$
Beispiel:
$$360 = 2*2*2*3*3*5$$ 

\subsection*{Aufgabe 0.3}
Satz \"uber Division mit Rest in den ganzen Zahlen:
$$\forall a \in \mathbb{Z} \; \forall b \in \mathbb{N} \; \exists! q, r \in \mathbb{Z}: a=q*b+r \enspace und \enspace 0 \le r < b .$$
Seien $a = 8 \in \mathbb{Z}$, $b = 5 \in \mathbb{N}$.
\\ Daraus folgt: $$a = q*b+r \Longleftrightarrow 8 = q*5+r \overset{0 \leq r < b}{\Longleftrightarrow} q=1 \enspace und \enspace r = 3.$$
Es gilt bei der Division mit Rest in den ganzen Zahlen bei der Wahl von a = 8 und b = 5 also: $$8 = 1*5+3.$$

% HIERUNTER WIEDER NICHTS VERAENDERN, SONST GIBT ES TECHNISCHE PROBLEME

\bigskip

\corr{korrigiert von \hspace{1cm} am }
\end{document}