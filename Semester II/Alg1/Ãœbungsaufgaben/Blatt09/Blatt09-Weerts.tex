\documentclass[12pt]{article}

\usepackage{a4}  
\usepackage{color}
\usepackage{amssymb}
\usepackage{amsmath}
\usepackage[utf8]{inputenc}
\usepackage{faktor}
 
\newcommand{\corr}[1]{\textcolor{red}{#1}}
\newcommand{\QED}{\begin{flushright} $\square$ \end{flushright}}
\newcommand{\QEDD}{\begin{flushright} "$\square$" \end{flushright}}
\newcommand{\df}{\enspace\Longrightarrow\enspace}
\newcommand{\koeff}[2]{\begin{pmatrix}#1 \\ #2\end{pmatrix}}
\newcommand{\koefff}[3]{\begin{pmatrix}#1 \\ #2 \\ #3\end{pmatrix}}
\newcommand{\Char}{\operatorname{char}}
\newcommand{\isIdeal}{\trianglelefteq}
\newcommand{\ann}{\operatorname{ann}}
\newcommand{\ideal}[1]{\langle#1\rangle}
\newcommand{\N}{\operatorname{N}}
\newcommand{\enorm}{\operatorname{d}}
\newcommand{\gdw}{\;\Longleftrightarrow\;}
\newcommand{\abs}[1]{\vert #1\vert}
\newcommand{\ggT}{\operatorname{ggT}}
\newcommand{\kgV}{\operatorname{kgV}}
\newcommand{\grad}{\operatorname{deg}}
\newcommand{\LC}{\operatorname{LC}}
\newcommand{\determinante}{\operatorname{det}}

\newcommand{\aal}{a_{\alpha}}
\newcommand{\ab}{a_{\beta}}
\newcommand{\ba}{b_{\alpha}}
\newcommand{\bb}{b_{\beta}}



\begin{document}
\section*{Abgabe Algebra I, Blatt 09}

Studierende(r): Weerts, Steffen, steffen.weerts@uni-oldenburg.de

\subsection*{Aufgabe 9.1}
\begin{enumerate}
	\item[(a)] Sei $p$ eine Primzahl, $F_p:=\sum_{i=0}^{p-1}t^i\in\mathbb{Q}[t]$ und $\zeta_p:=e^{\frac{2\pi i}{p}}$. \\
	Zu zeigen: $F_p$ ist Minimalpolynom von $\zeta_p$ über $\mathbb{Q}$. \\
	Es gilt:
	$$\left(e^{\frac{2\pi i}{p}}\right)^p=e^{2\pi i}=1.$$
	$$\df\text{für }h:=t^p-1\text{ gilt }h(\zeta_p)=0.$$
	Da jedoch $h$ nicht irreduzibel ist, ist $h$ nicht das Minimalpolynom von $\zeta_p$ über $\mathbb{Q}$. Es gilt:
	\begin{align*}
		t^p-1&=(t-1)(t^{p-1}+t^{p-2}+\cdots+t^2+t+1) \\
		&=\sum_{i=0}^{p-1}t^i=(t-1)\cdot F_p. \\
		\df e^{\frac{2\pi i}{p}}-1&=0\text{ oder }F_p/e^{\frac{2\pi i}{p}}=0.
	\end{align*}		
	Da $e^{\frac{2\pi i}{p}}-1=0\gdw p=1$, kann $e^{\frac{2\pi i}{p}}-1\neq0$, denn $p$ ist Primzahl und damit nicht $1$. Folglich muss $F_p(\zeta_p)=0$ gelten. \\
	Zu zeigen: $F_p$ ist irreduzibel. \\
	Es gilt:
	\begin{align*}
		&F_p=\sum_{i=0}^{p-1}t^i \\
		\df&F_p\text{ konstant}\gdw p=1 \\
		\overset{p\text{ Primzahl}}{\df}&F_p\text{ nicht konstant}.
	\end{align*}
	Sei $a:=1\in\mathbb{Q}$. Es gilt:
	\begin{align*}
		F_p(t+1)&=\sum_{i=0}^{p-1}(t+1)^i \\
		&=(t+1)^{p-1}+(t+1)^{p-2}+\cdots+(t+1)^2+(t+1)^1+(t+1)^0 \\
		&=\sum_{i=0}^{p-1}\koeff{p-1}{i}\cdot t^{p-1-i}\cdot1^i+\sum_{i=0}^{p-2}\koeff{p-2}{i}\cdot t^{p-2-i}\cdot1^i+\cdots \\
		&\quad\quad+\sum_{i=0}^{2}\koeff{2}{i}\cdot t^{2-i}\cdot1^i+\sum_{i=0}^{1}\koeff{1}{i}\cdot t^{1-i}\cdot1^i+\sum_{i=0}^{01}\koeff{0}{i}\cdot t^{0-i}\cdot1^i \\
	\end{align*}
	\begin{align*}
		&=\sum_{i=0}^{p-1}\left(\sum_{j=0}^{i}\koeff{i}{j}\cdot t^{i-j}\right) \\
	\end{align*}
	Die Koeffizienten der einzelnen Summen (für $n\in\{0,\cdots,p-1\})$) finden sich als Zeilen im Pascalschen Dreieck, wobei die Koeffizienten mit gleicher Potenz von $t$ auf einer Spalte (von rechts oben nach links unten) liegen. Diese Summe kann man also als Summe der Spalten bis zur Zeile $p-1$ auffassen. Diese ist:
	$$\sum_{i=0}^{p-1}\left(t^i\cdot\sum_{j=0}^{p-1}\koeff{j}{i}\right)=\sum_{i=0}^{p-1}\left(t^i\cdot\sum_{j=i}^{p-1}\koeff{j}{i}\right).$$
	Nach (A1) (siehe Anhang) gilt:
	$$\sum_{i=0}^{p-1}\left(t^i\cdot\sum_{j=i}^{p-1}\koeff{j}{i}\right)\overset{\text{A1}}{=}\sum_{i=0}^{p-1}\koeff{p}{i+1}\cdot t^i =:f$$
	Nun gilt:
	\begin{enumerate}
		\item[(i)] $p\nmid\koeff{p}{(p-1)+1}=1=\LC(F_p),$
		\item[(ii)] $p\overset{(A2)}{\mid}\koeff{p}{i+1}=a_i\quad\quad\forall i=0,\cdots,p-2,$
		\item[(iii)] $p^2\nmid\koeff{p}{1}=p=a_0.$
	\end{enumerate}
	$$\overset{\text{Eisenstein}}{\df}F_p\text{ Minimalpolynom von }\zeta_p\text{ über }\mathbb{Q}.$$
	
	Zu zeigen: $[\mathbb{Q}(\zeta_p):\mathbb{Q}]=p-1$ und $(1,\zeta_p,\zeta_p^2,\cdots,\zeta_p^{p-2})$ ist Basis von $\mathbb{Q}(\zeta_p)$. \\
	Es gilt:
	\begin{align*}
		&\zeta_p\text{ ist algebraisch über }\mathbb{Q} \\
		\overset{6.2.9}{\df}&[\mathbb{Q}(\zeta_p):\mathbb{Q}]=\grad(F_p)=p-1\text{ und }(1,\zeta_p,\zeta_p^2,\cdots,\zeta_p^{p-2})\text{ ist Basis von }\mathbb{Q}(\zeta_p).
	\end{align*}
	
	Zu zeigen: Alle Nullstellen von $F_p$ liegen in $\mathbb{Q}(\zeta_p)$. \\
	Fehlt.
	\QEDD
	
	
	\item[(b)] 
	\begin{enumerate}
		\item[(i)] Sei $f_1=t^4-5\in\mathbb{Q}[t]$. \\
		Es gilt:
		\begin{align*}
			f_1&=t^4-5 \\
			&=(t^2+\sqrt{5})(t^2-\sqrt{5}) \\
			&=(t+\sqrt[4]{5}i)(t-\sqrt[4]{5}i)(t+\sqrt[4]{5})(t-\sqrt[4]{5})
		\end{align*}
		Da $\pm\sqrt[4]{5},\pm\sqrt[4]{5}i\in\mathbb{Q}(\sqrt[4]{5},i)$, zerfällt $f_q$ über $\mathbb{Q}(\sqrt[4]{5},i)$. \\
		
		Zu zeigen: $[\mathbb{Q}(\sqrt[4]{5},i):\mathbb{Q}(\sqrt[4]{5})]=2$. \\
		Da $f_{i,\mathbb{Q}(\sqrt[4]{5})}=t^2+1$, ist $[\mathbb{Q}(\sqrt[4]{5},i):\mathbb{Q}(\sqrt[4]{5})]=\grad(f_{i,\mathbb{Q}(\sqrt[4]{5})})=2$. \\
		
		Zu zeigen: $[\mathbb{Q}(\sqrt[4]{5}):\mathbb{Q}]=4$. \\
		Angenommen, $\grad(f_{\sqrt[4]{5},\mathbb{Q}})=1$. \\
		Es gilt:
		\begin{align*}
			f_{\sqrt[4]{5},\mathbb{Q}}(\sqrt[4]{5})=\sqrt[4]{5}+a_0\overset{!}{=}0 \\
			\df a_0=-\sqrt[4]{5}\notin\mathbb{Q} \\
			\df \grad(f_{\sqrt[4]{5},\mathbb{Q}})\neq1.
		\end{align*}
		Angenommen, $\grad(f_{\sqrt[4]{5},\mathbb{Q}})=2$. \\
		Es gilt:
		\begin{align*}
			&f_{\sqrt[4]{5},\mathbb{Q}}(\sqrt[4]{5})=\sqrt[4]{5}^2+a_1\sqrt[4]{5}+a_0\overset{!}{=}0 \\
			\df &a_0=-\sqrt{5}-a_1\sqrt[4]{5}\notin\mathbb{Q}\quad\forall a_1\in\mathbb{Q} \\
			\df &\grad(f_{\sqrt[4]{5},\mathbb{Q}})\neq2.
		\end{align*}
		Angenommen, $\grad(f_{\sqrt[4]{5},\mathbb{Q}})=3$. \\
		Es gilt:
		\begin{align*}
			&f_{\sqrt[4]{5},\mathbb{Q}}(\sqrt[4]{5})=\sqrt[4]{5}^3+a_2\sqrt[4]{5}^2+a_1\sqrt[4]{5}+a_0 \\
			&\quad\quad\quad\quad=\sqrt{5}\sqrt[4]{5}+a_2\sqrt{5}+a_1\sqrt[4]{5}+a_0\overset{!}{=}0 \\
			\df &a_0=-\sqrt{5}\sqrt[4]{5}-a_2\sqrt{5}-a_1\sqrt[4]{5}\notin\mathbb{Q}\quad\forall a_1,a_2\in\mathbb{Q} \\
			\df &\grad(f_{\sqrt[4]{5},\mathbb{Q}})\neq3.
		\end{align*}
		Angenommen, $\grad(f_{\sqrt[4]{5},\mathbb{Q}})=4$. \\
		Es gilt:
		\begin{align*}
			&f_{\sqrt[4]{5},\mathbb{Q}}(\sqrt[4]{5})=\sqrt[4]{5}^4+a_3\sqrt[4]{5}^3+a_2\sqrt[4]{5}^2+a_1\sqrt[4]{5}+a_0 \\
			&\quad\quad\quad\quad=5+a_3\sqrt{5}\sqrt[4]{5}+a_2\sqrt{5}+a_1\sqrt[4]{5}+a_0\overset{!}{=}0 \\
			\df &a_0=-5-a_3\sqrt{5}\sqrt[4]{5}-a_2\sqrt{5}-a_1\sqrt[4]{5}\in\mathbb{Q}\text{ für }a_1=a_2=a_3=0 \\
			\df &f_{\sqrt[4]{5},\mathbb{Q}}=t^4-5 \\
			\df &[\mathbb{Q}(\sqrt[4]{5}):\mathbb{Q}]=\grad(f_{\sqrt[4]{5},\mathbb{Q}})=4.
		\end{align*}
		Insgesamt ergibt sich: $$[\mathbb{Q}(\sqrt[4]{5},i):\mathbb{Q}]=[\mathbb{Q}(\sqrt[4]{5},i):\mathbb{Q}(\sqrt[4]{5})]\cdot[\mathbb{Q}(\sqrt[4]{5}):\mathbb{Q}]=2\cdot4=8.$$
		\QED
		
		\item[(ii)] Sei $f_2=t^4+1$. \\
		Es gilt:
		\begin{align*}
			f_2&=t^4+1 \\
			&=(t^2+i)(t^2-i) \\
			&=(t+i\sqrt{i})(t-i\sqrt{i})(t+\sqrt{i})(t-\sqrt{i}).
		\end{align*}
		Da $\sqrt{i},i\sqrt{i}\notin\mathbb{Q}$, zerfällt $f_2$ nicht über $\mathbb{Q}$, jedoch über $\mathbb{Q}(i,\sqrt{i})$, denn  $\sqrt{i},i\sqrt{i}$ sind die Nullstellen von $f_2$ in $\mathbb{Q}(i,\sqrt{i})$. \\
		
		Zu zeigen: $[\mathbb{Q}(i):\mathbb{Q}]=2$. \\
		Es gilt:
		\begin{align*}
			&f_{i,\mathbb{Q}}=t^2+1. \\
			\df &[\mathbb{Q}(i):\mathbb{Q}]=\grad(f_{i,\mathbb{Q}})=2.
		\end{align*}
		
		Zu zeigen: $[\mathbb{Q}(i,\sqrt{i}):\mathbb{Q}(i)]=2$. \\
		Angenommen, $\grad(f_{\sqrt{i},\mathbb{Q}(i)})=1$. \\
		Es gilt:
		\begin{align*}
			f_{\sqrt{i},\mathbb{Q}(i)}(\sqrt{i})=\sqrt{i}+a_0\overset{!}{=}0 \\
			\df a_0=-\sqrt{i}\notin\mathbb{Q}(i) \\
			\df \grad(f_{\sqrt{i},\mathbb{Q}(i)})\neq1.
		\end{align*}
		Angenommen, $\grad(f_{\sqrt{i},\mathbb{Q}(i)})=2$. \\
		Es gilt:
		\begin{align*}
			&f_{\sqrt{i},\mathbb{Q}(i)}(\sqrt{i})=\sqrt{i}^2+a_1\sqrt{i}+a_0 \\
			&\quad\quad\quad=i+a_1\sqrt{i}+a_0\overset{!}{=}0 \\
			\df &a_0=-i-a_1\sqrt{i} \\
			\df &(a_0\in\mathbb{Q}(i)\gdw a_1=0) \\
			\df &f_{\sqrt{i},\mathbb{Q}(i)}=t^2-i \\
			\df &[\mathbb{Q}(i,\sqrt{i}):\mathbb{Q}(i)]=\grad(f_{\sqrt{i},\mathbb{Q}(i)})=2.
		\end{align*}
		Insgesamt ergibt sich: $$[\mathbb{Q}(i,\sqrt{i}):\mathbb{Q}]=[\mathbb{Q}(i,\sqrt{i}):\mathbb{Q}(i)]\cdot[\mathbb{Q}(i):\mathbb{Q}]=2\cdot2=4.$$
		\QED
	\end{enumerate}
	
	\item[(c)] 
	\begin{enumerate}
		\item[(i)] Sei $K:=\mathbb{Q}\subseteq\mathbb{Q}(i)=:L$ Körpererweiterung, $f=t^2+1\in K[t],n:=\grad(f)=2$. \\
		Es gilt:
		\begin{align*}
			f&=t^2+1 \\
			&=t^2-i^2 \\
			&=(t+i)(t-i).
		\end{align*}
		Außerdem gilt: $$t+i,t-i\in L[t]\df f\text{ zerfällt über }L\text{, aber nicht über K, da }i\notin K.$$		
		Zu zeigen: $[L:K]=2$. \\
		Da $i\notin\mathbb{Q}$ ist, muss der Grad des Minimalpolymoms von $i$ größer als $1$ sein. Da $i^2+1=0$, ist $f$ das Minimalpolynom von $i$ über $\mathbb{Q}$. \\
		Es gilt: $$[L:K]=\grad(f)=2.$$
		\QED
		
		\item[(ii)] Sei $K=\mathbb{R}\subseteq\mathbb{C}=\mathbb{R}(i)=L$ eine Körpererweiterung, $f=t^2+1\in\mathbb{R}[t]$ Polynom. \\
		Es gilt: $$f=t^2+1=(t+i)(t-i)$$
		Außerdem gilt:
		\begin{align*}
			&i\text{ algebraisch über }\mathbb{R} \\
			\df &[\mathbb{C}:\mathbb{Q}]=[\mathbb{Q}(i):\mathbb{Q}]=\grad(f)=2=2!.
		\end{align*}
		\QED
		
		\item[(iii)] Seien $K=\mathbb{Q}, L=\mathbb{Q}(\sqrt{2},i)$ Körper, $f:=t^3\in\mathbb{Q}[t]$ mit $n:=\grad(f)=3$. \\
		Es gilt:
		$$[L:K]=[\mathbb{Q}(\sqrt{2},i):\mathbb{Q}(\sqrt{2})]\cdot[\mathbb{Q}(\sqrt{2}):\mathbb{Q}]=2\cdot2=4$$
		Außerdem gilt:
		$$n=3<4=[L:K]=4<6=3!=n!$$
		Da $f$ über $\mathbb{Q}$ zerfällt, zerfällt $f$ auch über $\mathbb{Q}(\sqrt{2},i)$.
		\QED
	\end{enumerate}
\end{enumerate}

\subsection*{Aufgabe 9.2}
\begin{enumerate}
	\item[(a)] Sei $f:=t^4+t^3+2t^2+1\in\mathbb{Z}_3[t]$. \\
	Zu zeigen: $f=(t+1)\cdot(t^3+2t+1)$ ist eine Zerlegung von $f$ in irreduzible Polynome über $\mathbb{Z}_3$. \\
	Es gilt:
	\begin{align*}
		f&=t^4+t^3+2t^2+1 \\
		&=t^4+t^3+2t^2+2t+t+1 \\
		&=(t+1)(t^3+2t+1).
	\end{align*}
	Zu zeigen: $t+1$ irreduzibel über $\mathbb{Z}_3$. \\
	Es gilt: $$\mathbb{Z}_3\text{ Körper und }\grad(t+1)=1\overset{5.1.2}{\df} t+1\text{ irreduzibel über }\mathbb{Z}_3.$$
	Zu zeigen: $h:=t^3+2t+1$ irreduzibel über $\mathbb{Z}_3$. \\
	Angenommen, $h$ sei reduzibel über $\mathbb{Z}_3$. Da $\mathbb{Z}_3$ Körper ist und $\grad(h)=3$ hat $h$ nach Beobachtung 5.1.6 eine Nullstelle in $\mathbb{Z}_3$. Es gilt:
	\begin{align*}
		h(0)&=0^3+2\cdot0+1=1\neq0, \\
		h(1)&=1^3+2\cdot1+1=1\neq0, \\
		h(2)&=2^3+2\cdot2+1=1\neq0.
	\end{align*}
	Dies steht im Widerspruch zu Beobachtung 5.1.6, weshalb $h$ nicht reduzibel über $\mathbb{Z}_3$ sein kann. \\
	Insgesamt ergibt sich, dass $f=(t+1)\cdot(t^3+2t+1)$ eine Zerlegung von $f$ in irreduzible Polynome über $\mathbb{Z}_3$ ist.
	\QED
	
	\item[(b)] Fehlt.
	\item[(c)] Fehlt.
\end{enumerate}

\subsection*{Aufgabe 9.3}
Sei $R:=\mathbb{Z}_6$ und $M:=R\times R$. \\
Zu zeigen: $X:=((2,4))$ linear unabhängig. \\
Es gilt:
\begin{align*}
	&0\neq3\in R \\
	\df &3\cdot(2,4) = (3\cdot2,3\cdot4)=(0,0) \\
	\df &((2,4))\text{ ist eine linear abhängige Familie}.
\end{align*}
\QED

\subsection*{Anhang}
\begin{enumerate}
	\item[(A1)] Sei $n\in\mathbb{N}_0$, $p\in\mathbb{N}$. Dann ist $$\sum_{i=0}^{p-1}\koeff{n+i}{n}=\koeff{n+p}{n+1}.$$
	Beweis:
	\begin{enumerate}
		\item[(IA)] Sei $\underline{p=1}$. Es gilt:
		$$\sum_{i=0}^{1-1}\koeff{n+i}{n}=\koeff{n}{n}=\koeff{n+1}{n+1}=\koeff{n+p}{n+1}.$$
		
		\item[(IV)] Gelte die Behauptung für ein beliebiges, aber festes $p\in\mathbb{N}$.
		
		\item[(IS)] $\underline{p\leadsto p+1}$. \\
		Es gilt:
		\begin{align*}
			\sum_{i=0}^{(p+1)-1}\koeff{n+i}{n}&=\sum_{i=0}^{p-1}\koeff{n+i}{n}+\koeff{n+p}{n} \\
			&=\koeff{n+l}{n+1}+\koeff{n+p}{n}=\koeff{n+(p+1)}{n+1}.
		\end{align*}
	\end{enumerate}
	\QED
	
	\item[(A2)] Sei $p$ Primzahl. Sei $n\in\{1,2,\cdots,p-1\}$. Dann gilt: $$p\mid\koeff{p}{n}.$$
	 Beweis: \\
	 Sei $p$ Primzahl, $n\in\{1,2,\cdots,p-1\}$. Es gilt:
	 \begin{align*}
	 	&\koeff{p}{n}\cdot n!=\frac{p!}{n!\cdot(p-n)!}\cdot n! \\
	 	&\quad\quad\quad\quad=\frac{p!}{(p-n)!} \\
	 	&\quad\quad\quad\quad=p(p-1)(p-2)\cdots(p-n+2)(p-n+1) \\
	 	\df& p\mid\frac{p!}{(p-n)!} \\
	 	\df& p\mid\koeff{p}{n}\cdot n! \\
	 	\overset{p\text{ Primzahl}}{\underset{p>n>0}{\df}}&p\mid\koeff{p}{n}.
	 \end{align*}
	 \QED
\end{enumerate}


\bigskip

\corr{korrigiert von \hspace{1cm} am }
\end{document}