\documentclass[12pt]{article}

\usepackage{a4}  
\usepackage{color}
\usepackage{amssymb}
\usepackage{amsmath}
\usepackage[utf8]{inputenc}
\usepackage{faktor}
 
\newcommand{\corr}[1]{\textcolor{red}{#1}}
\newcommand{\QED}{\begin{flushright} $\square$ \end{flushright}}
\newcommand{\QEDD}{\begin{flushright} "$\square$" \end{flushright}}
\newcommand{\df}{\enspace\Longrightarrow\enspace}
\newcommand{\koeff}[2]{\begin{pmatrix}#1 \\ #2\end{pmatrix}}
\newcommand{\koefff}[3]{\begin{pmatrix}#1 \\ #2 \\ #3\end{pmatrix}}
\newcommand{\Char}{\operatorname{char}}
\newcommand{\isIdeal}{\trianglelefteq}
\newcommand{\ann}{\operatorname{ann}}
\newcommand{\ideal}[1]{\langle#1\rangle}
\newcommand{\N}{\operatorname{N}}
\newcommand{\enorm}{\operatorname{d}}
\newcommand{\gdw}{\;\Longleftrightarrow\;}
\newcommand{\abs}[1]{\vert #1\vert}
\newcommand{\ggT}{\operatorname{ggT}}
\newcommand{\kgV}{\operatorname{kgV}}
\newcommand{\grad}{\operatorname{deg}}
\newcommand{\LC}{\operatorname{LC}}
\newcommand{\determinante}{\operatorname{det}}

\newcommand{\aal}{a_{\alpha}}
\newcommand{\ab}{a_{\beta}}
\newcommand{\ba}{b_{\alpha}}
\newcommand{\bb}{b_{\beta}}



\begin{document}
\section*{Abgabe Algebra I, Blatt 08}

Studierende(r): Weerts, Steffen, steffen.weerts@uni-oldenburg.de

\subsection*{Aufgabe 8.1}
\begin{enumerate}
	\item[(a)] Sei $R$ faktorieller Ring mit $\Char(R)\neq2$. Sei $R[t_1,t_2]=(R[t_1])[t_2]$ Polynomring in zwei Variablen. \\
	Zu zeigen: $f=t_1^2+t_2^2+1\in R[t_1,t_2]$ ist irreduzibel in $R[t_1,t_2]$. \\
	Sei $a_0:=t_1^2+1\in R[t_1]$. Es gilt: $$f=t_2^2+t_1^2+1=t_2^2+a_0\in R[t_1,t_2].$$
	Da $R$ faktorieller Ring ist, sind nach dem Lemma von Gauß auch $R[t_1]$ und $R[t_1,t_2]$ faktorielle Ringe. \\
	Das Polynom $f$ ist primitiv, denn $f=t_2^2+(t_1^2+1)=1\cdot t_2^2+a_0\cdot t_2^0,\; a_0\in R[t_1]$ und somit ist $\ggT(1,0,a_0)=\ggT(1,\ggT(0,a_0))=\ggT(1,a_0)=1$. Dies gilt, da $f\in (R[t_1])[t_2]$ Koeffizienten aus dem Ring $R[t_1]$ besitzt. Der Grad von $f$ ist $\grad(f)=2$. \\
	Da $a_0=t_1^2+1$ irreduzibel ist, ist $a:=a_0=t_1^2+1$ prim in $R[t_1]$. \\
	Es gelten die Voraussetzungen:
	\begin{enumerate}
		\item[(i)] $a=t_1^2+1\nmid q=a_2$,
		\item[(ii)] $a=t_1^2+1\mid 0=a_1$, $a=t_1^2+1\mid t_1^2+1=a_0$,
		\item[(iii)] $a^2=(t_1^2+1)^2=t_1^4+2t_1^2+1\nmid t_1^2+1=a_0$.
	\end{enumerate}
	Daraus folgt nach dem Kriterium von Eisenstein, dass $f$ irreduzibel in $R[t_1,t_2]$ ist.
	\QED
	
	\item[(b)] Fehlt.
	
	\item[(c)] Fehlt.
\end{enumerate}

\subsection*{Aufgabe 8.2}
\begin{enumerate}
	\item[(a)] Sei $\alpha:=\frac{1-\sqrt[3]{6}}{2}\in\mathbb{R}$. \\
	Zu zeigen: $f_{\alpha,\mathbb{Q}}=$ ist Minimalpolynom von $\alpha$ über $\mathbb{Q}$. \\
	Angenommen, $\grad(f_{\alpha,\mathbb{Q}})=1$. Es gilt:
	\begin{align*}
		&f_{\alpha,\mathbb{Q}}=\alpha+a_0\overset{!}{=}0 \\
		\df&a_0=-\alpha\notin\mathbb{Q} \\
		\df&\grad(f_{\alpha,\mathbb{Q}})\neq1.
	\end{align*}
	
	Angenommen, $\grad(f_{\alpha,\mathbb{Q}})=2$. Es gilt:
	\begin{align*}
		&f_{\alpha,\mathbb{Q}}=\alpha^2+a_1\alpha+a_0=\left(\frac{1-\sqrt[3]{6}}{2}\right)^2+a_1\alpha+a_0 \\
		&\quad\quad\quad\quad=\left(\frac{1}{2}\right)^2-2\frac{1}{2}\frac{\sqrt[3]{6}}{2}+\left(\frac{\sqrt[3]{6}}{2}\right)^2+a_1\alpha+a_0 \\
		&\quad\quad\quad\quad=\frac{1}{4}-\frac{\sqrt[3]{6}}{2}+\frac{\sqrt[3]{6}^2}{4}+a_1\alpha+a_0 \\
		&\quad\quad\quad\quad=\frac{1}{4}-\frac{\sqrt[3]{6}}{2}+\frac{\sqrt[3]{6}^2}{4}+a_1\frac{1}{2}+a_1\frac{\sqrt[3]{6}}{2}+a_0\overset{!}{=}0 \\
		\df&a_0=-\frac{1}{4}-\frac{a_2}{2}-(a_1-1)\frac{\sqrt[3]{6}}{2}-\frac{\sqrt[3]{6}^2}{4}\notin\mathbb{Q} \\
		\df&\grad(f_{\alpha,\mathbb{Q}})\neq2.
	\end{align*}
	
	Angenommen, $\grad(f_{\alpha,\mathbb{Q}})=3$. Es gilt:
	\begin{align*}
		&f_{\alpha,\mathbb{Q}}=\alpha^3+a_2\alpha^2+a_1\alpha+a_0 \\
		&\quad\quad\quad\quad=\left(\frac{1-\sqrt[3]{6}}{2}\right)^3+a_2\alpha^2+a_1\alpha+a_0 \\
		&\quad\quad\quad\quad=\left(\frac{1}{2}\right)^3-3\left(\frac{1}{2}\right)^2\frac{\sqrt[3]{6}}{2}+3\frac{1}{2}\left(\frac{\sqrt[3]{6}}{2}\right)^2-\left(\frac{\sqrt[3]{6}}{2}\right)^3 \\
		&\quad\quad\quad\quad\quad\quad+a_2\alpha^2+a_1\left(\frac{1-\sqrt[3]{6}}{2}\right)+a_0 \\
		&\quad\quad\quad\quad=\frac{1}{8}-\frac{3}{4}\frac{\sqrt[3]{6}}{2}+\frac{3}{2}\left(\frac{\sqrt[3]{6}}{2}\right)^2-\frac{6}{8}+a_2\alpha^2+\frac{a_1}{2}-a_1\frac{\sqrt[3]{6}}{2}+a_0 \\
		&\quad\quad\quad\quad=\frac{1}{8}-\frac{6}{8}+\frac{a_1}{2}+a_0+\left(-a_1-\frac{3}{4}\right)\frac{\sqrt[3]{6}}{2}+\frac{3}{2}\left(\frac{\sqrt[3]{6}}{2}\right)^2+a_2\alpha^2 \\
		&\quad\quad\quad\quad=-\frac{5}{8}+\frac{a_1}{2}+a_0+\left(-a_1-\frac{3}{4}\right)\frac{\sqrt[3]{6}}{2}+\frac{3}{2}\left(\frac{\sqrt[3]{6}}{2}\right)^2+a_2\left(\frac{1-\sqrt[3]{6}}{2}\right)^2 \\
		&\quad\quad\quad\quad=-\frac{5}{8}+\frac{a_1}{2}+a_0+\left(-a_1-\frac{3}{4}\right)\frac{\sqrt[3]{6}}{2}+\frac{3}{2}\left(\frac{\sqrt[3]{6}}{2}\right)^2 \\
		&\quad\quad\quad\quad\quad\quad+a_2\left(\left(\frac{1}{2}\right)^2-2\frac{1}{2}\frac{\sqrt[3]{6}}{2}+\left(\frac{\sqrt[3]{6}}{2}\right)^2\right) \\
	\end{align*}
	\begin{align*}		
		&\quad\quad\quad\quad=-\frac{5}{8}+\frac{a_2}{4}+\frac{a_1}{2}+a_0+\left(-a_1-\frac{3}{4}-a_2\right)\frac{\sqrt[3]{6}}{2}+\left(\frac{3}{2}+a_2\right)\left(\frac{\sqrt[3]{6}}{2}\right)^2\overset{!}{=}0 \\
		\df&a_2=-\frac{3}{2},a_1=\frac{3}{4},a_0=\frac{5}{8} \\
		\df&f_{\alpha,\mathbb{Q}}=t^3-\frac{3}{2}+\frac{3}{4}+\frac{5}{8}\text{ ist Minimalpolynom von $\alpha$ über $\mathbb{Q}$}.
	\end{align*}
	
	Da $\grad(f_{\alpha,\mathbb{Q}})=3$ gilt, folgt $$\left[\mathbb{Q}(\alpha):\mathbb{Q}\right]=\grad(f_{\alpha,\mathbb{Q}})=3.$$
	\QED
	
	\item[(b)] Sei $\alpha:=\sqrt{5}+i\in\mathbb{C}$ und $K=\mathbb{Q}(\sqrt{5},i)$. \\
	Zu zeigen: $f_{\alpha,\mathbb{Q}}=t^4-8t^2+36$ ist Minimalpolynom von $\alpha$ über $\mathbb{Q}$. \\
	Es gilt: $$\left[\mathbb{Q}(\sqrt{5},i):\mathbb{Q}\right]=\left[\mathbb{Q}(\sqrt{5},i):\mathbb{Q}(\sqrt{5})\right]\cdot\left[\mathbb{Q}(\sqrt{5}):\mathbb{Q}\right]$$
	Bestimme zunächst $\left[\mathbb{Q}(\sqrt{5},i):\mathbb{Q}(\sqrt{5})\right]$. \\
	Angenommen, $\grad(f_{i,\mathbb{Q}(\sqrt{5})})=1$. Es gilt:
	\begin{align*}
		&f_{i,\mathbb{Q}(\sqrt{5})}(i)=i+a_0\overset{!}{=}0 \\
		\df&a_0=-i\notin\mathbb{Q}(\sqrt{5}) \\
		\df&\grad(f_{i,\mathbb{Q}(\sqrt{5})})\neq1.
	\end{align*}
	
	Angenommen, $\grad(f_{i,\mathbb{Q}(\sqrt{5})})=2$. Es gilt:
	\begin{align*}
		&f_{i,\mathbb{Q}(\sqrt{5})}(i)=i^2+a_1i+a_0=a_1i+a_0-1\overset{!}{=}0 \\
		\df&\left(a_0=1-a_1i\notin\mathbb{Q}(\sqrt{5})\gdw a_1=0\right) \\
		\df&f_{i,\mathbb{Q}(\sqrt{5})}=t^2+1,\;\grad(f_{i,\mathbb{Q}(\sqrt{5})})=2.
	\end{align*}
	Da $\mathbb{Q}(\sqrt{5})$ ein Körper ist, gilt $\left[\mathbb{Q}(\sqrt{5},i):\mathbb{Q}(\sqrt{5})\right]=\grad(f_{i,\mathbb{Q}(\sqrt{5})})=2$.
	Bestimme nun $\left[\mathbb{Q}(\sqrt{5}):\mathbb{Q}\right]$. \\
	Angenommen, $\grad(f_{\sqrt{5},\mathbb{Q}})=1$. Es gilt:
	\begin{align*}
		&f_{\sqrt{5},\mathbb{Q}}(\sqrt{5})=\sqrt{5}+a_0\overset{!}{=}0 \\
		\df&a_0=-\sqrt{5}\notin\mathbb{Q} \\
		\df&\grad(f_{\sqrt{5},\mathbb{Q}})\neq1.
	\end{align*}
	Angenommen, $\grad(f_{\sqrt{5},\mathbb{Q}})=2$. Es gilt:
	\begin{align*}
		&f_{\sqrt{5},\mathbb{Q}}(\sqrt{5})=\sqrt{5}^2+a_1\sqrt{5}+a_0=a_1\sqrt{5}+a_0+5\overset{!}{=}0 \\
		\df&\left(a_0=-a_1\sqrt{5}-5\in\mathbb{Q}\gdw a_1=0\right) \\
		\df&f_{\sqrt{5},\mathbb{Q}}=t^2-5,\;\grad(f_{\sqrt{5},\mathbb{Q}})=2.
	\end{align*}
	Da $\mathbb{Q}$ ein Körper ist, gilt $\left[\mathbb{Q}(\sqrt{5}):\mathbb{Q}\right]=\grad(f_{\sqrt{5},\mathbb{Q}})=2$. \\
	Also ergibt sich 
	\begin{align*}
		&\left[\mathbb{Q}(\sqrt{5},i):\mathbb{Q}\right]=\left[\mathbb{Q}(\sqrt{5},i):\mathbb{Q}(\sqrt{5})\right]\cdot\left[\mathbb{Q}(\sqrt{5}):\mathbb{Q}\right] \\
		=&\,\grad(f_{i,\mathbb{Q}(\sqrt{5})})\cdot\grad(f_{\sqrt{5},\mathbb{Q}}) \\
		=&\,2\cdot2 \\
		=&\,4.
	\end{align*}
	Daraus folgt, dass das Minimalpolynom $f_{\alpha,\mathbb{Q}}$ einen Grad von $\grad(f_{\alpha,\mathbb{Q}})=\left[\mathbb{Q}(\sqrt{5},i):\mathbb{Q}\right]=4$ besitzt. \\
	Daher existieren $a_0,a_1,a_2,a_3$ so, dass $f_{\alpha,\mathbb{Q}}=\alpha^4+a_3\alpha^3+a_2\alpha^2+a_1\alpha+a_0=0$ gilt.
	Es gilt:
	\begin{align*}
		\alpha&=\sqrt{5}+i, \\
		\alpha^2&=(\sqrt{5}+i)^2 \\
		&=\sqrt{5}^2+2\sqrt{5}i+i^2 \\
		&=4+2\sqrt{5}i, \\
		\alpha^3&=(\sqrt{5}+i)^3 \\
		&=\sqrt{5}^3+3\sqrt{5}^2i+3\sqrt{5}i^2+i^3 \\
		&=5\sqrt{5}+15i-3\sqrt{5}-i \\
		&=2\sqrt{5}+14i, \\
		\alpha^4&=(\sqrt{5}+i)^4 \\
		&=\sqrt{5}^4+4\sqrt{5}^3i+6\sqrt{5}^2i^2+4\sqrt{5}i^3+i^4 \\
		&=25+20\sqrt{5}i-30-4\sqrt{5}i+1 \\
		&=-4+16\sqrt{5}i. \\
	\end{align*}
	\begin{align*}		
		f_{\alpha,\mathbb{Q}}&=\alpha^4+a_3\alpha^3+a_2\alpha^2+a_1\alpha+a_0 \\
		&=-4+16\sqrt{5}i+a_3(2\sqrt{5}+14i)+a_2(4+2\sqrt{5}i)+a_1(\sqrt{5}+i)+a_0 \\
		&=-4+4a_2+a_0+2a_3\sqrt{5}+a_1\sqrt{5}+16\sqrt{5}i+14a_3i+2a_2\sqrt{5}i+a_1i \\
		&=-4+4a_2+a_0+(2a_3+a_1)\sqrt{5}+(16\sqrt{5}+14a_3+2a_2\sqrt{5}+a_1)i \\
		&=-4+4a_2+a_0+(2a_3+a_1)\sqrt{5}+(14a_3+a_1+(16+2a_2)\sqrt{5})i \\
		&\overset{!}{=}0.
	\end{align*}
	Damit $f_{\alpha,\mathbb{Q}}=0$ gilt, müssen folgende Gleichungen erfüllt sein:
	\begin{enumerate}
		\item[(I)] $-4+4a_2+a_0=0$.
		\item[(II)] $2a_3+a_1=0$.
		\item[(III)] $14a_3+a_1=0$.
		\item[(IV)] $16+2a_2=0$.\\
		
		\item[ad (IV):] $16+2a_2=0\gdw a_2=-8$.
		\item[ad (I):] $-4+4a_2+a_0=-4+4\cdot(-8)+a_0=0\gdw a_0=36$.
		\item[ad (II),(III):] (II) und (III)$\gdw a_1=a_3=0$.
	\end{enumerate}
	Daraus folgt, dass $f_{\alpha,\mathbb{Q}}=t^4-8t^2+36\in\mathbb{Q}[t]$ gelten muss. \\
	
	Zu zeigen: $(1,\sqrt{5},i,\sqrt{5}i)$ ist Basis von $\mathbb{Q}(\alpha)$. \\
	Es gilt:
	\begin{align*}
		&(1,\sqrt{5})\text{ ist Basis von }\mathbb{Q}(\sqrt{5})\text{ bezüglich }\mathbb{Q} \\
		&\text{ und }(1,i)\text{ ist Basis von }\mathbb{Q}(\sqrt{5},i)\text{ bezüglich }\mathbb{Q}(\sqrt{5}) \\
		\df&\forall a\in\mathbb{Q}(\sqrt{5})\exists x,y\in\mathbb{Q}:x\cdot1+y\cdot\sqrt{5}=a \\
		&\text{ und }\forall b\in\mathbb{Q}(\sqrt{5},i)\exists x,y\in\mathbb{Q}(\sqrt{5}):x\cdot1+y\cdot i=b \\
		\df&\forall c\in\mathbb{Q}(\sqrt{5},i)\exists w,x,y,z\in\mathbb{Q}:(w\cdot1+x\cdot\sqrt{5})\cdot1+(y\cdot1+z\cdot\sqrt{5})\cdot i=c \\
		\df&\forall c\in\mathbb{Q}(\sqrt{5},i)\exists w,x,y,z\in\mathbb{Q}:w+x\cdot\sqrt{5}+y\cdot i+z\cdot\sqrt{5}i=c \\
		\df&(1,\sqrt{5},i,\sqrt{5}i)\text{ ist Basis von }\mathbb{Q}(\sqrt{5},i).
	\end{align*}
	\QED
\end{enumerate}

\subsection*{Aufgabe 8.3}
\begin{enumerate}
	\item[(a)] Sei $p\in\mathbb{N}$ eine Primzahl und $\alpha:=\sqrt{p+\sqrt{p}}\in\mathbb{R}$. \\
	Zu zeigen: $f_{\alpha,\mathbb{Q}}=t^4-2pt^2+p^2-p$ ist Minimalpolynom von $\alpha$ über $\mathbb{Q}$. \\
	Prüfe nun, welchen Grad $f_{\alpha,\mathbb{Q}}$ besitzt, indem überprüft wird, ob $\sum_{i=0}^na_i\alpha^i$ rational ist. Der Grad des Minimalpolynoms ist das niedrigste $n\in\mathbb{N}$, für das $\sum_{i=0}^na_i\alpha^i$ rational ist.
	
	Angenommen, $\grad(f_{\alpha,\mathbb{Q}})=1$. Es gilt:
	$$f_{\alpha,\mathbb{Q}}=\alpha+a_0\overset{!}{=}0\overset{\alpha\notin\mathbb{Q}}{\df}\grad(f_{\alpha,\mathbb{Q}})\neq1.$$
	
	Angenommen, $\grad(f_{\alpha,\mathbb{Q}})=2$. Es gilt:
	\begin{align*}
		f_{\alpha,\mathbb{Q}}&=\alpha^2+a_1\alpha+a_0 \\
		&=\sqrt{p+\sqrt{p}}^2+a_1\sqrt{p+\sqrt{p}}+a_0 \\
		&=p+\sqrt{p}+a_1\sqrt{p+\sqrt{p}}+a_0 \\
		\overset{\sqrt{p}+a_1\alpha\notin\mathbb{Q}}{\df}&\grad(f_{\alpha,\mathbb{Q}})\neq2.
	\end{align*}
	
	Angenommen, $\grad(f_{\alpha,\mathbb{Q}})=3$. Es gilt:
	\begin{align*}
		f_{\alpha,\mathbb{Q}}&=\alpha^3+a_2\alpha^2+a_1\alpha+a_0 \\
		&=\sqrt{p+\sqrt{p}}^3+a_2\sqrt{p+\sqrt{p}}^2+a_1\sqrt{p+\sqrt{p}}+a_0 \\
		&=(p+\sqrt{p})\cdot\sqrt{p+\sqrt{p}}+a_2(p+\sqrt{p})+a_1\sqrt{p+\sqrt{p}}+a_0 \\
		&=p\cdot\sqrt{p+\sqrt{p}}+\sqrt{p}\cdot\sqrt{p+\sqrt{p}}+a_2p+a_2\sqrt{p}+a_1\sqrt{p+\sqrt{p}}+a_0 \\
		\overset{(p+\sqrt{p})\alpha\notin\mathbb{Q}}{\df}&\grad(f_{\alpha,\mathbb{Q}})\neq3.
	\end{align*}
	
	Angenommen, $\grad(f_{\alpha,\mathbb{Q}})=3$. Es gilt:
	\begin{align*}
		f_{\alpha,\mathbb{Q}}&=\alpha^4+a_3\alpha^3+a_2\alpha^2+a_1\alpha+a_0 \\
		&=\sqrt{p+\sqrt{p}}^4+a_3\sqrt{p+\sqrt{p}}^3+a_2\sqrt{p+\sqrt{p}}^2+a_1\sqrt{p+\sqrt{p}}+a_0 \\
		&=(p+\sqrt{p})^2+a_3(p+\sqrt{p})\sqrt{p+\sqrt{p}}+a_2(p+\sqrt{p})+a_1\sqrt{p+\sqrt{p}}+a_0 \\
		&=p^2+2p\sqrt{p}+\sqrt{p}^2+a_3(p+\sqrt{p})\sqrt{p+\sqrt{p}}+a_2p+a_2\sqrt{p}+a_1\sqrt{p+\sqrt{p}}+a_0 \\
		&=p^2+p+a_2p+a_0+2p\sqrt{p}+a_2\sqrt{p}+a_3(p+\sqrt{p})\sqrt{p+\sqrt{p}}+a_1\sqrt{p+\sqrt{p}} \\
		&=p^2+p+a_2p+a_0+(2p+a_2)\sqrt{p}+(a_3(p+\sqrt{p})+a_1)\sqrt{p+\sqrt{p}} \\
		&\overset{!}{=}0.
	\end{align*}
	Um diese Gleichung zu lösen, kann man ein lineares Gleichungssystem verwenden:
	\begin{enumerate}
		\item[(I)] $p^2+p+a_2p+a_0=0$,
		\item[(II)] $2p+a_2=0$,
		\item[(III)] $a_3(p+\sqrt{p})+a_1=0$.
	\end{enumerate}
	\begin{enumerate}
		\item[ad (II):] $2p+a_2=0\gdw a_2=-2p$.
		\item[ad (I):] $p^2+p+a_2p+a_0=p^2+p+(-2p)p+a_0=0\gdw a_0=p^2-p$.
		\item[ad (III):] $a_3(p+p\sqrt{p})+a_1=0\overset{\sqrt{p}\notin\mathbb{Q}}{\gdw}a_1=a_3=0$.
	\end{enumerate}
	Setzt man nun die berechneten Werte von $a_0,a_1,a_2,a_3$ in die Gleichung $f_{\alpha,\mathbb{Q}}=\alpha^4+a_3\alpha^3+a_2\alpha^2+a_1\alpha+a_0$ ein und ersetzt $\alpha$ durch die Variable $t$, so erhält man das Minimalpolynom
	$$f_{\alpha,\mathbb{Q}}=t^4-2pt^2+p^2-p.$$
	
	Zu zeigen: $\left[\mathbb{Q}(\alpha):\mathbb{Q}\right]=4$. \\
	Es gilt: $$\left[\mathbb{Q}(\alpha):\mathbb{Q}\right]\overset{\alpha\text{ alg. über }\mathbb{Q}}{=}\left[\mathbb{Q}[\alpha]:\mathbb{Q}\right]=\grad(f_{\alpha,\mathbb{Q}})=4.$$
	
	Zu zeigen: $(1,\alpha,\alpha^2,\alpha^3)$ ist Basis von $\mathbb{Q}(\alpha)$. \\
	Da $\alpha$ algebraisch über $\mathbb{Q}$ mit Minimalpolynom $f_{\alpha,\mathbb{Q}}$ vom Grad 4 ist, gilt nach Korollar 6.2.9, dass $(1,\alpha,\alpha^2,\alpha^3)$ Basis von $\mathbb{Q}(\alpha)$ ist.
	
	Zu zeigen: $\alpha^{-1}$. \\
	Fehlt.
	
	Zu zeigen: $\alpha^5=14\alpha^3-42\alpha$. \\
	Es gilt:
	\begin{align*}
		\alpha&=\sqrt{7+\sqrt{7}}, \\
		\alpha^2&=\sqrt{7+\sqrt{7}}^2 \\
		&=7+\sqrt{7}, \\•
		\alpha^3&=\sqrt{7+\sqrt{7}}^3 \\
		&=\sqrt{7+\sqrt{7}}^2\cdot\alpha \\
		&=(7+\sqrt{7})\cdot\alpha \\
		&=7\alpha+\sqrt{7}\alpha.
	\end{align*}
	Nun gilt:
	\begin{align*}
		\alpha^5&=\alpha^2\alpha^2\alpha \\
		&=(7+\sqrt{7})\cdot(7+\sqrt{7})\cdot\alpha \\
		&=(49+14\sqrt{7}+\sqrt{7}^2)\cdot\alpha \\
		&=56\alpha+14\sqrt{7}\alpha \\
		&=56\alpha-14\cdot7\alpha+14\cdot7\alpha+14\sqrt{7}\alpha \\
		&=(56-98)\cdot\alpha+14\alpha^3 \\
		&=14\alpha^3-42\alpha.
	\end{align*}
	\QEDD
	
	\item[(b)] Fehlt.
\end{enumerate}


\bigskip

\corr{korrigiert von \hspace{1cm} am }
\end{document}